% ---
% Capa
% ---
\imprimircapa
% ---

% ---
% Folha de rosto
% (o * indica que haverá a ficha bibliográfica)
% ---
\imprimirfolhaderosto*
% ---

% ---
% Inserir a ficha bibliografica
% ---
% http://ficha.bu.ufsc.br/
\begin{fichacatalografica}
	\includepdf{beforetext/Ficha_Catalografica.pdf}
\end{fichacatalografica}
% ---

% ---
% Inserir folha de aprovação
% ---
\begin{folhadeaprovacao}
	\OnehalfSpacing
	\centering
	\imprimirautor\\%
	\vspace*{10pt}		
	\textbf{\imprimirtitulo}%
	\ifnotempty{\imprimirsubtitulo}{:~\imprimirsubtitulo}\\%
	%		\vspace*{31.5pt}%3\baselineskip
	\vspace*{\baselineskip}
	%\begin{minipage}{\textwidth}
	% ~do~\imprimirprograma~do~\imprimircentro~da~\imprimirinstituicao~para~a~obtenção~do~título~de~\imprimirformacao.
	Este~\imprimirtipotrabalho~foi julgado adequado para obtenção do Título de “\imprimirformacao” e aprovado em sua forma final pelo~\imprimirprograma. \\
		\vspace*{\baselineskip}
	\imprimirlocal, \imprimirdata. \\
	\vspace*{2\baselineskip}
	\assinatura{\OnehalfSpacing\imprimircoordenador \\ \imprimircoordenadorRotulo~do Curso}
	\vspace*{2\baselineskip}
	\textbf{Banca Examinadora:} \\
	\vspace*{\baselineskip}
	\assinatura{\OnehalfSpacing\imprimirorientador \\ \imprimirorientadorRotulo}
	%\end{minipage}%
	\vspace*{\baselineskip}
	\assinatura{Me. Eduardo Willwock Lussi,\\
	Avaliador \\
	Universidade Federal de Santa Catarina}

	\vspace*{\baselineskip}
	% TODO: Descobrir o sobrenome do Obama
	\assinatura{Ma. Letícia Bertuzzi\\
	Avaliador(a) \\
	Universidade Federal de Santa Catarina}


\end{folhadeaprovacao}
% ---

% ---
% Dedicatória
% ---
%\begin{dedicatoria}
%	\vspace*{\fill}
%	\noindent
%	\begin{adjustwidth*}{}{5.5cm}     
%		Este trabalho é dedicado aos meus colegas de classe e aos meus queridos pais.
%	\end{adjustwidth*}
%\end{dedicatoria}
% ---

% ---
% Agradecimentos
%
%\begin{agradecimentos}
%	Inserir os agradecimentos aos colaboradores à execução do trabalho. 
%	
%	Xxxxxxxxxxxxxxxxxxxxxxxxxxxxxxxxxxxxxxxxxxxxxxxxxxxxxxxxxxxxxxxxxxxxxx. 
%\end{agradecimentos}
% ---

% ---
% Epígrafe
% ---
%\begin{epigrafe}
%	\vspace*{\fill}
%	\begin{flushright}
%		\textit{``Texto da Epígrafe.\\
%			Citação relativa ao tema do trabalho.\\
%			É opcional. A epígrafe pode também aparecer\\
%			na abertura de cada seção ou capítulo.\\
%			Deve ser elaborada de acordo com a NBR 10520.''\\
%			(Autor da epígrafe, ano)}
%	\end{flushright}
%\end{epigrafe}
% ---

% ---
% RESUMOS
% ---

% resumo em português
\setlength{\absparsep}{18pt} % ajusta o espaçamento dos parágrafos do resumo
\begin{resumo}
	Atualmente, o desenvolvimento de algoritmos quânticos é limitado pela falta de formas eficazes e acessíveis de executá-los na prática. Hardware quântico moderno ainda apresenta baixa capacidade de processamento e seu uso é caro. Dessa forma, é adequada a simulação do algoritmo em hardware clássico para auxiliar no desenvolvimento dos algoritmos e reduzir o custo desse processo. Para isso, existe o Ket, uma plataforma para desenvolvimento de algoritmos quânticos que permite que o usuário simule localmente seus programas.

Por sua vez, a simulação clássica também traz desafios: o consumo de memória cresce exponencialmente em relação ao número de qubits do circuito a ser simulado, tornando-a inviável para circuitos maiores. O objetivo deste trabalho é implementar um novo método de simulação e integrá-lo ao Ket. Esse novo método uniliza o QMDD (Quantum Multiple-valued Decision Diagram), uma estrutura de dados capaz de representar e executar circuitos quânticos por meio de matrizes, explorando a redundância para codificá-las de forma a mitigar o consumo exponencial de memória previamente mencionado.

\textbf{Palavras-chave}: Computação Quântica. Simulação Quântica. Diagrama de Decisão. Ket.
\end{resumo}

% resumo em inglês
\begin{resumo}[Abstract]
	\SingleSpacing
	\begin{otherlanguage*}{english}
		% TODO
		\textbf{Keywords}: Quantum Computing. Quantum Simulation. Decision Diagram. Ket.
	\end{otherlanguage*}
\end{resumo}

%% resumo em francês 
%\begin{resumo}[Résumé]
% \begin{otherlanguage*}{french}
%    Il s'agit d'un résumé en français.
% 
%   \textbf{Mots-clés}: latex. abntex. publication de textes.
% \end{otherlanguage*}
%\end{resumo}
%
%% resumo em espanhol
%\begin{resumo}[Resumen]
% \begin{otherlanguage*}{spanish}
%   Este es el resumen en español.
%  
%   \textbf{Palabras clave}: latex. abntex. publicación de textos.
% \end{otherlanguage*}
%\end{resumo}
%% ---

{%hidelinks
	\hypersetup{hidelinks}
	% ---
	% inserir lista de ilustrações
	% ---
	\pdfbookmark[0]{\listfigurename}{lof}
	\listoffigures*
	\cleardoublepage
	% ---
	
	% ---
	% inserir lista de quadros
	% ---
	\pdfbookmark[0]{\listofquadrosname}{loq}
	\listofquadros*
	\cleardoublepage
	% ---
	
	% ---
	% inserir lista de tabelas
	% ---
	\pdfbookmark[0]{\listtablename}{lot}
	\listoftables*
	\cleardoublepage
	% ---
	
	% ---
	% inserir lista de abreviaturas e siglas (devem ser declarados no preambulo)
	% ---
	\imprimirlistadesiglas
	% ---
	
	% ---
	% inserir lista de símbolos (devem ser declarados no preambulo)
	% ---
	\imprimirlistadesimbolos
	% ---
	
	% ---
	% inserir o sumario
	% ---
	\pdfbookmark[0]{\contentsname}{toc}
	\tableofcontents*
	\cleardoublepage
	
}%hidelinks
% ---
