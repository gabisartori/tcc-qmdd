% ------------------------------------------------------------------------
% ------------------------------------------------------------------------
% Modelo UFSC para Trabalhos Academicos (tese de doutorado, dissertação de
% mestrado) utilizando a classe abntex2
%
% Autor: Alisson Lopes Furlani
% 	Modificações:
%	- 27/08/2019: Alisson L. Furlani, add pacote 'glossaries' para listas
%   - 06/11/2019: Luiz-Rafael Santos, modifica para Trabalho de Conclusão de Curso
% ------------------------------------------------------------------------
% ------------------------------------------------------------------------

\documentclass[
	% -- opções da classe memoir --
	12pt,				% tamanho da fonte
	%openright,			% capítulos começam em pág ímpar (insere página vazia caso preciso)
	oneside,			% para impressão no anverso. Oposto a twoside
	a4paper,			% tamanho do papel. 
	% -- opções da classe abntex2 --
	chapter=TITLE,		% títulos de capítulos convertidos em letras maiúsculas
	section=TITLE,		% títulos de seções convertidos em letras maiúsculas
	%subsection=TITLE,	% títulos de subseções convertidos em letras maiúsculas
	%subsubsection=TITLE,% títulos de subsubseções convertidos em letras maiúsculas
	% -- opções do pacote babel --
	english,			% idioma adicional para hifenização
	%french,				% idioma adicional para hifenização
	%spanish,			% idioma adicional para hifenização
	brazil				% o último idioma é o principal do documento
	]{abntex2}

\usepackage{setup/ufscthesisA4-alf}

% ---
% Filtering and Mapping Bibliographies
% ---
% Pacotes de citações
% ---
\usepackage{csquotes}
\usepackage[backend = biber, style = abnt]{biblatex}
% FIXME Se desejar estilo numérico de citações,  comente a linha acima e descomente a linha a seguir.
% \usepackage[backend = biber, style = numeric-comp]{biblatex}

\setlength\bibitemsep{\baselineskip}
\DeclareFieldFormat{url}{Disponível~em:\addspace\url{#1}}
\NewBibliographyString{sineloco}
\NewBibliographyString{sinenomine}
\DefineBibliographyStrings{brazil}{%
	sineloco     = {\mkbibemph{S\adddot l\adddot}},
	sinenomine   = {\mkbibemph{s\adddot n\adddot}},
	andothers    = {\mkbibemph{et\addabbrvspace al\adddot}},
	in			 = {\mkbibemph{In:}}
}

\addbibresource{aftertext/references.bib} % Seus arquivos de referências

% ---
\DeclareSourcemap{
	\maps[datatype=bibtex]{
		% remove fields that are always useless
		\map{
			\step[fieldset=abstract, null]
			\step[fieldset=pagetotal, null]
		}
		% remove URLs for types that are primarily printed
%		\map{
%			\pernottype{software}
%			\pernottype{online}
%			\pernottype{report}
%			\pernottype{techreport}
%			\pernottype{standard}
%			\pernottype{manual}
%			\pernottype{misc}
%			\step[fieldset=url, null]
%			\step[fieldset=urldate, null]
%		}
		\map{
			\pertype{inproceedings}
			% remove mostly redundant conference information
			\step[fieldset=venue, null]
			\step[fieldset=eventdate, null]
			\step[fieldset=eventtitle, null]
			% do not show ISBN for proceedings
			\step[fieldset=isbn, null]
			% Citavi bug
			\step[fieldset=volume, null]
		}
	}
}
% ---

% ---
% Informações de dados para CAPA e FOLHA DE ROSTO
% ---
\autor{Gabriel Sartori Rangel}
\titulo{Uso de Diagramas de Decisão para Simulação de Circuitos Quânticos no Ket}
% \subtitulo{Subtítulo (se houver)}
% orientador.
\orientador{Me. Evandro Chagas Ribeiro da Rosa}
\coorientador[coorientadora]{Profa. Dra. Jerusa Marchi}
\coordenador[coordenadora]{Profa. Dra. Lúcia Helena Martins Pacheco}
\ano{2026}
% FIXME Substituir '[dia] de [mês] de [ano]' pela data em que ocorreu sua defesa.
\data{[dia] de [mês] de [ano]}
\local{Florianópolis}
\instituicaosigla{UFSC}
\instituicao{Universidade Federal de Santa Catarina}
\tipotrabalho{Trabalho de Conclusão de Curso}
\formacao{bacharel em ciência da computação}
\nivel{bacharel}
\programa{Curso de Graduação em Ciência da Computação}
\centro{Centro Tecnológico}
\preambulo
{%
\imprimirtipotrabalho~do~\imprimirprograma~do~\imprimircentro~da~\imprimirinstituicao~para~a~obtenção~do~título~de~\imprimirformacao.
}
% ---

% ---
% Configurações de aparência do PDF final
% ---
% alterando o aspecto da cor azul
\definecolor{blue}{RGB}{41,5,195}
% informações do PDF
\makeatletter
\hypersetup{
     	%pagebackref=true,
		pdftitle={\@title}, 
		pdfauthor={\@author},
    	pdfsubject={\imprimirpreambulo},
	    pdfcreator={LaTeX with abnTeX2},
		pdfkeywords={ufsc, latex, abntex2}, 
		colorlinks=true,       		% false: boxed links; true: colored links
    	linkcolor=black,%blue,          	% color of internal links
    	citecolor=black,%blue,        		% color of links to bibliography
    	filecolor=black,%magenta,      		% color of file links
		urlcolor=black,%blue,
		bookmarksdepth=4
}
\makeatother
% ---

% ---
% compila a lista de abreviaturas e siglas e a lista de símbolos
% ---

% Declaração das siglas
\siglalista{ABNT}{Associação Brasileira de Normas Técnicas}

% Declaração dos simbolos
\simbololista{C}{\ensuremath{C}}{Circunferência de um círculo}
\simbololista{pi}{\ensuremath{\pi}}{Número pi} 
\simbololista{r}{\ensuremath{r}}{Raio de um círculo}
\simbololista{A}{\ensuremath{A}}{Área de um círculo}

% compila a lista de abreviaturas e siglas e a lista de símbolos
\makenoidxglossaries 

% ---

% ---
% compila o indice
% ---
\makeindex
% ---

% ----
% Início do documento
% ----
\begin{document}

% Seleciona o idioma do documento (conforme pacotes do babel)
%\selectlanguage{english}
\selectlanguage{brazil}

% Retira espaço extra obsoleto entre as frases.
\frenchspacing 

% Espaçamento 1.5 entre linhas
\OnehalfSpacing

% Corrige justificação
%\sloppy

% ----------------------------------------------------------
% ELEMENTOS PRÉ-TEXTUAIS
% ----------------------------------------------------------
% \pretextual %a macro \pretextual é acionado automaticamente no início de \begin{document}
% ---
% Capa, folha de rosto, ficha bibliografica, errata, folha de apróvação
% Dedicatória, agradecimentos, epígrafe, resumos, listas
% ---
% ---
% Capa
% ---
\imprimircapa
% ---

% ---
% Folha de rosto
% (o * indica que haverá a ficha bibliográfica)
% ---
\imprimirfolhaderosto*
% ---

% ---
% Inserir a ficha bibliografica
% ---
% http://ficha.bu.ufsc.br/
\begin{fichacatalografica}
  \includepdf{beforetext/Ficha_Catalografica.pdf}
\end{fichacatalografica}
% ---

% ---
% Inserir folha de aprovação
% ---
\begin{folhadeaprovacao}
  \OnehalfSpacing
  \centering
  \imprimirautor\\%
  \vspace*{10pt}    
  \textbf{\imprimirtitulo}%
  \ifnotempty{\imprimirsubtitulo}{:~\imprimirsubtitulo}\\%
  %    \vspace*{31.5pt}%3\baselineskip
  \vspace*{\baselineskip}
  %\begin{minipage}{\textwidth}
  % ~do~\imprimirprograma~do~\imprimircentro~da~\imprimirinstituicao~para~a~obtenção~do~título~de~\imprimirformacao.
  Este~\imprimirtipotrabalho~foi julgado adequado para obtenção do Título de “\imprimirformacao” e aprovado em sua forma final pelo~\imprimirprograma. \\
    \vspace*{\baselineskip}
  \imprimirlocal, \imprimirdata. \\
  \vspace*{2\baselineskip}
  \assinatura{\OnehalfSpacing\imprimircoordenador \\ \imprimircoordenadorRotulo~do Curso}
  \vspace*{2\baselineskip}
  \textbf{Banca Examinadora:} \\
  \vspace*{\baselineskip}
  \assinatura{\OnehalfSpacing\imprimirorientador \\ \imprimirorientadorRotulo}
  \assinatura{\OnehalfSpacing\imprimircoorientador \\ \imprimircoorientadorRotulo}
  %\end{minipage}%
  \vspace*{\baselineskip}
  \assinatura{Me. Eduardo Willwock Lussi,\\
  Avaliador \\
  Universidade Federal de Santa Catarina}

  \vspace*{\baselineskip}
  \assinatura{Ma. Letícia Bertuzzi\\
  Avaliadora \\
  Universidade Federal de Santa Catarina}
\end{folhadeaprovacao}
% ---

% ---
% Dedicatória
% ---
%\begin{dedicatoria}
%  \vspace*{\fill}
%  \noindent
%  \begin{adjustwidth*}{}{5.5cm}     
%    Este trabalho é dedicado aos meus colegas de classe e aos meus queridos pais.
%  \end{adjustwidth*}
%\end{dedicatoria}
% ---

% ---
% Agradecimentos
%
%\begin{agradecimentos}
%  Inserir os agradecimentos aos colaboradores à execução do trabalho. 
%  
%  Xxxxxxxxxxxxxxxxxxxxxxxxxxxxxxxxxxxxxxxxxxxxxxxxxxxxxxxxxxxxxxxxxxxxxx. 
%\end{agradecimentos}
% ---

% ---
% Epígrafe
% ---
%\begin{epigrafe}
%  \vspace*{\fill}
%  \begin{flushright}
%    \textit{``Texto da Epígrafe.\\
%      Citação relativa ao tema do trabalho.\\
%      É opcional. A epígrafe pode também aparecer\\
%      na abertura de cada seção ou capítulo.\\
%      Deve ser elaborada de acordo com a NBR 10520.''\\
%      (Autor da epígrafe, ano)}
%  \end{flushright}
%\end{epigrafe}
% ---

% ---
% RESUMOS
% ---

% resumo em português

\setlength{\absparsep}{18pt} % ajusta o espaçamento dos parágrafos do resumo
\begin{resumo}
  Atualmente, o desenvolvimento de algoritmos quânticos é limitado pela falta de formas eficazes e acessíveis de executá-los na prática. Hardware quântico moderno ainda apresenta baixa capacidade de processamento e seu uso é caro. Dessa forma, é adequada a simulação do algoritmo em hardware clássico para viabilizar processo. Para isso, existe o Ket, uma plataforma para elaboração de algoritmos quânticos que permite que o usuário simule localmente seus programas.

  Por sua vez, a simulação quântica em computadores clássicos também traz desafios: nos métodos convencionais, o consumo de memória cresce exponencialmente em relação ao número de qubits do circuito a ser simulado, tornando-a inviável para circuitos maiores. O objetivo deste trabalho é implementar um novo método de simulação e integrá-lo ao Ket. Esse método se baseia na técnica de simulação de circuitos por meio da decomposição em matrizes que representam cada porta lógica do circuito. Para contornar o crescimento exponencial das matrizes, este trabalho utiliza o QMDD ({\itshape Quantum Multiple-valued Decision Diagram}), uma estrutura de dados capaz de compactar matrizes com alta redundância em seus valores. Essas matrizes são comumente encontradas em circuitos com baixa taxa de entrelaçamento, permitindo que eles sejam simulados em espaço e tempo menor que exponencial.

  \textbf{Palavras-chave}: Computação Quântica. Simulação Quântica. Diagrama de Decisão. Ket.
\end{resumo}

% resumo em inglês
\begin{resumo}[Abstract]
  \begin{otherlanguage*}{english}
    Currently, the development of quantum algorithms is limited by the lack of efficient means to execute them. Modern quantum hardware still presents low capacity and its usage is expensive. Therefore, it is adequate to simulate these algorithms via classic hardware in order to make this process viable. Hence there is Ket, a platform of quantum development, which allows its user to simulate locally their quantum programs.

    Similarly, quantum simulation in classic computers of quantum algorithms also has impediments: memory usage for conventional methods grows exponentially in relation to the number of qubits in the circuit, rendering it infeasible for larger circuits. This work's goal is to implement a new simulation method and integrate it to Ket. This method is based on the simulation technique of decomposing the circuit in matrices that represent each of the circuit's logic gates. In order to mitigate the exponential growth in the size of said matrices, this work will make use of QMDD (Quantum Multiple-valued Decision Diagram), a data structure capable of compressing matrices with a high level of redundance in its values. These matrices are commonly found in circuits with low levels of entanglement, allowing for them to be simulated in space and time less than exponential.

    \textbf{Keywords}: Quantum Computing. Quantum Simulation. Decision Diagram. Ket.
  \end{otherlanguage*}
\end{resumo}

%% resumo em francês 
%\begin{resumo}[Résumé]
% \begin{otherlanguage*}{french}
%    Il s'agit d'un résumé en français.
% 
%   \textbf{Mots-clés}: latex. abntex. publication de textes.
% \end{otherlanguage*}
%\end{resumo}
%
%% resumo em espanhol
%\begin{resumo}[Resumen]
% \begin{otherlanguage*}{spanish}
%   Este es el resumen en español.
%  
%   \textbf{Palabras clave}: latex. abntex. publicación de textos.
% \end{otherlanguage*}
%\end{resumo}
%% ---

{%hidelinks
  \hypersetup{hidelinks}
  % ---
  % inserir lista de ilustrações
  % ---
  \pdfbookmark[0]{\listfigurename}{lof}
  \listoffigures*
  \cleardoublepage
  % ---
  
  % ---
  % inserir lista de quadros
  % ---
  % \pdfbookmark[0]{\listofquadrosname}{loq}
  % \listofquadros*
  % \cleardoublepage
  % ---
  
  % ---
  % inserir lista de tabelas
  % ---
  % \pdfbookmark[0]{\listtablename}{lot}
  % \listoftables*
  % \cleardoublepage
  % ---
  
  % ---
  % inserir lista de abreviaturas e siglas (devem ser declarados no preambulo)
  % ---
  \imprimirlistadesiglas
  % ---
  
  % ---
  % inserir lista de símbolos (devem ser declarados no preambulo)
  % ---
  \imprimirlistadesimbolos
  % ---
  
  % ---
  % inserir o sumario
  % ---
  \pdfbookmark[0]{\contentsname}{toc}
  \tableofcontents*
  \cleardoublepage
  
}%hidelinks
% ---

% ---

% ----------------------------------------------------------
% ELEMENTOS TEXTUAIS
% ----------------------------------------------------------
\textual

% ---
% 1 - Introdução
% ---
\chapter{Introdução}\label{chap:intro}

Conforme os fenômenos da mecânica quântica foram melhor descritos e compreendidos, um novo paradigma para solução de problemas surge: a computação quântica. Essa nova forma de elaborar algoritmos é equivalente à computação clássica em termos de computabilidade \cite{shor1998quantum}, porém apresenta vantagem em termos de complexidade computacional para certos problemas. Por exemplo, dada uma lista de items sem organização prévia, na computação clássica, a melhor forma de verificar se determinado item está presente nessa lista é percorrendo-a, elemento por elemento, buscando pelo alvo. Essa solução possui complexidade temporal O(n), o que significa que o tempo para encontrar um elemento específico cresce linearmente em relação ao tamanho da lista (uma lista 4 vezes maior que outra leva 4 vezes mais tempo para ser percorrida). Porém, Grover demonstrou que, usando da mecânica quântica, é possível resolver esse mesmo problema com uma complexidade temporal O($\sqrt{n}$) \cite{grover}. Nesse caso, uma lista 4 vezes maior que outra leva apenas o dobro do tempo para que o elemento desejado seja encontrado. Assim, sabe-se que é possível que algoritmos quânticos superem algoritmos clássicos em desempenho, nesse caso, como a complexidade foi reduizada de $O(n)$ para $O(\sqrt{n})$, diz-se que houve um ganho quadrático de desempenho.

Atualmente, porém, essa área de pesquisa encontra um obstáculo: a dificuldade de executar os algoritmos elaborados, afetando significativamente o desenvolvimento de novos algoritmos, já que testá-los torna-se uma tarefa manual e demorada sem um computador para realizá-la. Essa dificuldade existe porque hardware que possibilite a execução das operações fundamentais da computação quântica é caro e dificilmente acessível. Além disso, mesmo quando acessível, hardware quântico moderno apresenta baixa capacidade, permitindo apenas a execução de algoritmos simples sobre exemplos de casos pequenos \cite{golec2024quantum}.

% Substituir "favoráveis" por algo mais explicativo
Dessa forma, para auxiliar no desenvolvimento de algoritmos quânticos, busca-se solução na simulação por meio de computadores clássicos para compensar a falta de hardware quântico. Visto que ambas as formas de definir algoritmos são computacionalmente equivalentes, sabe-se que todo algoritmo quântico pode ser executado em uma máquina clássica e vice-versa. Então, basta definir uma forma de converter algoritmos quânticos em algoritmos clássicos e executar essa nova versão em uma máquina clássica, cujo custo e capacidade são altamente mais favoráveis para o desenvolvimento. Neste trabalho, essa conversão será feita decompondo o circuito em matrizes, cada porta lógica é representada por uma matriz e a execução do algoritmo se dá pelo produto das matrizes \cite{nielsen2010quantum}.

Por sua vez, a simulação quântica também apresenta desafios. Devido à capacidade dos computadores quânticos de sobrepor um número exponencial de estados possíveis em relação à sua capacidade física, a simulação precisa ser capaz de armazenar uma quantia exponencial de informação, tornando o consumo de memória inviável. De forma a mitigar esse problema, diferentes técnicas de simulação surgem, tentando otimizar o consumo de memória e tempo, como a simulação esparsa \cite{jaques2022leveraging} e o método de tableau \cite{aaronson2004improved}.

Este trabalho consiste na implementação de uma dessas técnicas: o uso de diagramas de decisão para simulação de circuitos quânticos. Nessa técnica, o circuito a ser simulado tem suas portas lógicas convertidas em matrizes e a execução do circuito é representada pela multiplicação sucessiva dessas matrizes. Porém, como mencionado previamente, o tamanho da informação a ser armazenada cresce exponencialmente em relação à largura do circuito, esse crescimento é refletido nas matrizes. Por isso o QMDD (Quantum Multiple-Value Decision Diagram) é útil, essa estrutura de dados permite representar matrizes de forma concisa, explorando a redundância na composição delas \cite{fujita1997multi}. Essa técnica de compressão apresenta ganho variado, pois matrizes com maior redundância serão melhor compactadas que matrizes de menor redundância. Por exemplo, uma matriz 1000x1000 na qual todos os valores são 1 pode ser descrita exatamente dessa forma "uma matriz 1000x1000 preenchida inteiramente com o valor 1" ao invés de dedicar espaço suficiente para armazenar \num{1000000} de números 1's um ao lado do outro. Já na \cref{distinctmatrix}, todos seus valores são diferentes, logo o QMDD não será capaz de compactá-la com a mesma eficácia, consumindo mais memória.

\begin{equation}
  \label{distinctmatrix}
  \begin{bmatrix}
    1 & 2 & \cdots & 1000 \\
    1001 & 1002 & \cdots & 2000 \\
    \vdots & \vdots & \ddots & \vdots \\
    999001 & 999002 & \cdots & 1000000
  \end{bmatrix}
\end{equation}


% TODO: Fazer uma chamada para esse parágrafo, mencionando que a possibilidade de representar matrizes de forma concisa permite a execução de circuitos de maneira mais eficiente que a existente no ket.
Assim, será feita uma implementação dos algoritmos necessários para simulação de um circuito quântico, envolvendo a conversão de cada porta em um QMDD que representa a matriz equivalente à porta, o produto entre dois QMDDs e a medição sobre determinado QMDD. Essa implementação então será integrada à plataforma Ket \cite{evandro}, uma interface feita para facilitar a descrição de algoritmos quânticos. Então, por meio do módulo de simulação por QMDD, o usuário poderá simular de forma eficiente algoritmos que não exijam matrizes de baixa redundância.

Além disso, não é intuitivamente claro quais circuitos levarão a uma matriz com alta ou baixa redundância. Então, este trabalho buscará comparar o desempenho de diferentes circuitos de forma a analisar se existem aspectos em comum entre os circuitos que apresentaram ganho significativo.

\section{Objetivos}

\subsection{Objetivo Geral}

Implementar um novo método de simulação para o Ket, que poderá ser escolhido pelo usuário para otimizar a simulação de circuitos cujas representações matriciais possuam grande redundância.

\subsection{Objetivos Específicos}

\begin{enumerate}[label=O\arabic*]
  \item Compilar e descrever os algoritmos necessários para o QMDD;

  \item Implementar um simulador utilizando os algoritmos do artigo;

  \item Integrar o simulador produzido ao código fonte do Ket;

  \item Identificar em quais cenários o simulador produzido apresenta vantagem em relação aos métodos de simulação já disponíveis na plataforma.
\end{enumerate}

% TODO: Estrutura do texto
\section {Estrutura do Documento}

\section{Trabalhos Correlatos}

% TODO: Mencionar o trabalho 
Existem dois artigos que descrevem o funcionamento do QMDD e oferecem implementações do algoritmos como exemplos e também para exibir o consumo de memória para circuitos pequenos. Nenhuma implementação foi em Rust e também há poucas conclusões sobre os tipos de circuitos nos quais o QMDD apresenta vantagem ou não.

\section{Metodologia}

Este trabalho apresenta a construção de um simulador a partir de um método de simulação ainda não utilizado pela plataforma Ket. A construção desse simulador será feita por meio da compilação de algoritmos existentes na literatura de forma cumprir todos os requisitos do Ket para simulação de circuitos quânticos.

Os resultados da implementação serão analisados de forma qualiquantitativa. Será feita uma comparação das estatísticas produzidas pela análise de desempenho do simulador implementado e dos simuladores já disponíveis no Ket de forma a verifcar que existem cenários nos quais o novo simulador apresenta vantagem em relação aos já existentes. Além disso, será análisado em quais cenários o novo simulador apresentou vantagem e em quais o ganho não foi significativo para identificar aspectos em comum dos algoritmos quânticos que puderam ser simulados de forma eficaz por meio do método apresentado. 

% ---

% ---
% 2 - Capítulo 2
% ---
\chapter{Computação Quântica}\label{cap:compq}
% TODO: Motivar a existência desse capítulo como sendo uma introdução à computação quântica para entendimento do funcionamento e utilidade do QMDD. Mencionar que é o livro Nielsen Chuang é recomendado para uma abordagem mais aprofundada.


A computação quântica é um paradigma para elaboração de algoritmos que se baseia nos fenômenos da mecânica quântica, como a superposição e o entrelaçamento de estados. Existem diferentes modelos de computação quântica como a computação adiabática \cite{guarienti2016computaccao} e a computação circuital. Neste trabalho, as propriedades quânticas de um sistema serão representadas por meio da computação circuital. Assim, todo sistema é representado por um circuito, que é composto de qubits e portas lógicas quânticas. Esse modelo representa sistemas quânticos por meio de espaços vetoriais de números complexos, nos quais cada posição do vetor representa um dos possíveis resultados a serem obtidos ao final da execução de um algoritmo \cite{nielsen2010quantum}.

\section{Qubits}

Análogo ao bit da computação clássica, o qubit é a menor unidade de informação de um sistema quântico. Para a computação quântica de circuitos, um qubit é representado por um vetor em $\mathbb{C}^2$. Diferente do bit clássico que se encontra sempre completamente em 0 ou em 1, o qubit pode so encontrar em uma superposição de estados, ou seja, é possível que o resultado da medição seja tanto 0 quanto 1. Por isso, é importante representá-lo como uma combinação dessas duas possibilidades, pois o qubit pode estar em 0 e 1 ao mesmo tempo. Por sua vez, essa superposição pode não ser perfeitamente equilibrada: um qubit pode se encontrar mais próximo de um estado que outro. Então, os 2 números complexos que representam o qubit indicam o quão próximo o qubit está de cada estado.

Para representar essa informação textualmente, pode-se utilizar a notação bra-ket, na qual vetores que representam estados comuns recebem símbolos específicos para representá-los. Por exemplo o \ket{0} (lê-se "ket zero") e o \ket{1} (ket um) são conhecidos como estados da base computacional e representam os seguintes vetores:
\begin{equation}
  \ket{0} = \begin{bmatrix}
    1 \\
    0
  \end{bmatrix}
  \qquad
  \ket{1} = \begin{bmatrix}
    0 \\
    1
  \end{bmatrix}
\end{equation}

Isso significa que \ket{0} está totalmente em 0 e \ket{1} totalmente em 1. Uma superposição arbitrária \ket{\psi} pode, então, ser representada como uma soma ponderada de \ket{0} e \ket{1}

\begin{equation}
  \ket{\psi}
  =
  \alpha  \cdot  \ket{0} + \beta  \cdot  \ket{1}
  =
  \begin{bmatrix}
    \alpha \\
    \beta
  \end{bmatrix}
\end{equation}

sendo $\alpha$ e $\beta$ valores complexos tais que $|\alpha|^2 + |\beta|^2 = 1$. O Motivo dessa restrição e o efeito que esses valores possuem no resultado da computação serão explicados na próxima seção.

\subsection{Medição e Colapso da Superposição}\label{sub:result}

% TODO: Mencionar que a operação de medida retorna um valor.
% Provavelmente vou ter que reescrever esse parágrafo em 2, um falando do processo de medida e outro falando do colapso.
A superposição de estados do qubit não é uma informação que possa ser extraída de um circuito quântico. Ao tentar observar o estado de um qubit em superposição, ele colapsa em \ket{0} ou \ket{1}. Esse colapso é aleatório, mas a probabilidade de cada resultado pode ser alterada: um qubit no estado $\ket{\psi} = \alpha \cdot \ket{0}+\beta \cdot \ket{1}$ possui $|\alpha|^2$ de chance de colapsar para \ket{0} e $|\beta|^2$ de chance de colapsar em \ket{1}. Por isso, um vetor $\begin{bmatrix} \alpha \\ \beta \end{bmatrix}$ só será um estado quântico válido se a soma do quadrado dos módulos de $\alpha$ e $\beta$ resultarem em 1, ou seja, 100\% de chance que o qubit colapse em 0 ou em 1 \cite{griffiths1995introduction}.

\subsection{Fase}
Como a probabilidade de determinado resultado é dada pelo quadrado do módulo do valor que representa esse resultado no vetor do qubit, multiplicar esse valor por qualquer valor complexo de módulo 1 não alterará a medição imediata de um estado. Porém ainda é possível que a evolução temporal do estado seja alterada. A \cref{sec:gates} oferece um exemplo desse caso.

Esse fator complexo recebe o nome de fase e pode ser divido em fase global e fase relativa. A fase global é quando todos os estados da superposição possuem um múltiplo em comum e não altera nem a medição nem a evolução temporal do qubit. Isso se dá pelo fato que todas as transformações da evolução temporal são lineares, então esse múltiplo em comum pode ser apenas fatorado.

Já a fase relativa é o nome dado para quando nem todos os estados da superposição possuem o mesmo múltiplo. Por exemplo no caso $\frac{1}{\sqrt{2}}\cdot(\ket{0} + (-1)\cdot \ket{1})$, \ket{0} é multiplicado por 1 e \ket{1} é multiplicado por -1 então diz-se que existe uma fase relativa de valor -1 sobre estado \ket{1}.

\subsubsection{Esfera de Bloch}

Como a fase global não afeta o qubit, é possível representar um qubit $\begin{bmatrix} \alpha \\ \beta\end{bmatrix}$ por meio de uma figura tridimensional. Duas dessas dimensões são ocupadas pelas partes real e imaginária de $\beta$ enquanto $\alpha$ pode ser sempre simplificado para um número real positivo, precisando apenas de uma dimensão para ser representado. Por exemplo, o estado $\frac{1}{\sqrt{2}}\cdot(i \cdot \ket{0} - \ket{1})$ pode ter $i$ fatorado como fase global, sendo equivalente ao estado $\frac{1}{\sqrt{2}}\cdot(\ket{0} + i \cdot \ket{1})$.

Essa forma de transformar $\mathbb{C}^2$ em um espaço tridimensional ao descartar a fase global pode ser visualizada por meio da Esfera de Bloch. Ao longo da latitude da esfera encontra-se a proporção entre \ket{0} e \ket{1} e ao longo da longitude encontra-se a fase relativa, atribuída como um valor de módulo 1 multiplicando a proporção de \ket{1}.

Na \cref{fig:bloch}, o polo norte da esfera representa um qubit completamente em \ket{0} enquanto o polo sul representa um qubit em \ket{1}. Ao longo do equador, estão todos os estados nos quais o qubit tenha a mesma chance de colapsar para \ket{0} ou \ket{1} ao ser medido, diferindo apenas pela fase, indicada pelo ângulo da longitude do ponto. Dessa forma, todos os estados sobre o equador da esfera seguem o formato $\frac{1}{\sqrt{2}}\cdot(\ket{0} + e^{i\theta}\cdot\ket{1})$ sendo $\theta$ o ângulo longitudinal da posição na esfera \cite{glendinning2005bloch}.

\begin{figure}[h]
  \centering
  \includegraphics[width=0.5\linewidth]{images/bloch-sphere.png}
  \caption{Esfera de Bloch. Fonte: \cite{wiki:bloch}}
  \label{fig:bloch}
\end{figure}

Os kets $\ket{+}, \ket{+i}, \ket{-} e \ket{-i}$, presentes no equador da esfera, representam 4 estados comuns nos quais $\theta$ vale $0, \pi/2, \pi, 3*pi/2$, respectivamente.


\section{Portas Lógicas}\label{sec:gates}

Visto que um qubit pode ser representado por meio de um vetor, operações sobre qubits podem ser representadas por meio de matrizes. De forma a respeitar o espaço de estados válidos para um qubit, é importante que a matriz que codifica a transformação possua determinante de valor 1. Isso é importante pois uma matriz unitária não altera a norma dos vetores aos quais ela é aplicada. Para a representação do qubit, isso significa que a soma das probabilidades de ambos os resultados possíveis não será alterada \cite{nielsen2010quantum}.

% TODO: apagar esse parágrafo ig?
Além disso, por possuirem determinante diferente de 0, matrizes unitárias são sempre reversíveis. Isso faz com que todo circuito quântico seja reversível, já que todos os seus componentes também são. Dessa forma, computadores quânticos se apresentam mais eficientes energeticamente, já que uma das principais formas de dissipação de energia que ocorre em computadores clássicos se dá pela destruição de informação que acontece em portas lógicas não reversíveis. Por exemplo, a porta lógica AND, que recebe 2 bits como entrada mas retorna apenas 1 de saída. Assim, diferentes entradas são forçadas a compartilhar a mesma saída, tornando impossível inferir qual a entrada da porta tendo como base apenas o resultado \cite{efficientreversiblegates}.

Um exemplo de matriz próximo da computação clássica é a matriz de Pauli X \cite{nielsen2010quantum}.

\begin{equation}
  X = \begin{bmatrix}
    0 & 1 \\
    1 & 0
  \end{bmatrix}
\end{equation}

Essa porta quântica basicamente inverte os coeficientes de \ket{0} e \ket{1}, atuando similarmente à porta NOT da computação clássica. Essa porta pode também agir sobre uma superposição. Considere, por exemplo, o estado $\ket{\psi} = 0.6 \cdot \ket{0} + 0.8 \cdot \ket{1}$. Nesse estado, ao ser medido, o qubit possui $36\% (|0.6|^2)$ de chance de colapsar em \ket{0} e $64\% (|0.8|^2)$ de colapsar em \ket{1}.

\begin{equation}
  \begin{bmatrix}
    0 & 1 \\
    1 & 0
  \end{bmatrix}
  \cdot
  \begin{bmatrix}
    0.6 \\
    0.8
  \end{bmatrix}
  =
  \begin{bmatrix}
    0.8 \\
    0.6
  \end{bmatrix}
\end{equation}

Após a aplicação da porta X, o qubit agora possui 64\% de chance de colapsar em \ket{0} e 36\% de chance de colapsar em \ket{1}.

Outra porta quântica bastante comum e útil é a porta de Hadamard, representada pela letra H.
\begin{equation}
  H
  =
  \frac{1}{\sqrt{2}}
  \cdot
  \begin{bmatrix}
    1 & 1 \\
    1 & -1
  \end{bmatrix}
\end{equation}

Ao aplicar essa porta em \ket{0}, o qubit é levado para uma superposição na qual existe 50\% de chance do qubit ser medido em \ket{0} e 50\% de chance de ser medido em \ket{1}. Note que o mesmo aconteceria se a porta fosse aplicada em \ket{1}, mas os resultados são diferentes. Enquanto $H \cdot\ket{0} = \frac{1}{\sqrt{2}}\cdot(\ket{0} + \ket{1})$, $H \cdot \ket{1} = \frac{1}{\sqrt{2}} \cdot (\ket{0} - \ket{1})$. Isso é um exemplo de como um qubit pode ter a sua fase alterada por meio de operações quânticas. Apesar de não alterar o resultado imediato de uma medição, a fase de um qubit pode alterar o resultado da superposição após a aplicação de outras portas quânticas. Por exemplo, nos estados recém mencionados, apesar de que apenas a fase seja diferente, aplicar a porta H neles novamente irá produzir resultados diferentes:
\begin{equation}
  H \begin{bmatrix}
    \frac{1}{\sqrt{2}} \\
    \frac{1}{\sqrt{2}}
  \end{bmatrix}
  =
  \begin{bmatrix}
    1 \\
    0
  \end{bmatrix}
  \qquad
  H \begin{bmatrix}
    \frac{1}{\sqrt{2}} \\
    -\frac{1}{\sqrt{2}}
  \end{bmatrix}
  =
  \begin{bmatrix}
    0 \\
    1
  \end{bmatrix}
\end{equation}

\subsection{Portas Quânticas Relevantes}

Algumas portas quânticas serão utilizadas como exemplos ao longo deste trabalho. Além disso, a matriz identidade é essencial para a definição e funcionamento da representação de circuitos quânticos por meio de matrizes. Esta seção lista e descreve o funcionamento dessas matrizes.

A matriz identidade, representada pela letra I. Essa matriz não altera o estado do qubit.

\begin{equation}
  I = \begin{bmatrix}
    1 & 0 \\
    0 & 1
  \end{bmatrix}
\end{equation}

Matrizes de Pauli: são 3 matrizes que descrevem as transformações análogas às rotações nos eixos X, Y e Z da esfera de bloch. Por isso são representadas pelas letras X, Y e Z \cite{nielsen2010quantum}.

\begin{equation}
  X = \begin{bmatrix}
    0 & 1 \\
    1 & 0
  \end{bmatrix}
\end{equation}

\begin{equation}
  Y = \begin{bmatrix}
    0 & -i \\
    i & 0
  \end{bmatrix}
\end{equation}

\begin{equation}
  Z = \begin{bmatrix}
    1 & 0 \\
    0 & -1
  \end{bmatrix}
\end{equation}

% TODO: Introduzir matrizes de fase S e T
Matrizes de fase S e T:

\begin{equation}
  S = \begin{bmatrix}
    1 & 0 \\
    0 & i
  \end{bmatrix}
\end{equation}

\begin{equation}
  T = \begin{bmatrix}
    1 & 0 \\
    0 & e^{\pi/2}
  \end{bmatrix}
\end{equation}

Matriz de Hadamard, coloca o qubit em superposição. Representada pela letra H.

\begin{equation}
  H = \frac{1}{\sqrt{2}} \begin{bmatrix}
    1 & 1 \\
    1 & -1
  \end{bmatrix}
\end{equation}

% TODO: apresentar porta CNOT
Além disso, é possível que portas atuem sobre múltiplos qubits. Um exemplo é a porta CNOT, que age sobre 2 qubits, um é o controle e o outro é o alvo. 

\begin{equation}
  CNOT = \begin{bmatrix}
    1 & 0 & 0 & 0 \\
    0 & 1 & 0 & 0 \\
    0 & 0 & 0 & 1 \\
    0 & 0 & 1 & 0
  \end{bmatrix}
  \label{eq:cnot}
\end{equation}

Note que, por atuar sobre mais de um qubit, a matriz dessa operação tem formato 4x4 e não 2x2. A seção a seguir apresenta o método para representar e manipular circuitos com múltiplos qubits.

\section{Circuitos com Múltiplos Qubits}

Para combinar múltiplos qubits em um único sistema, é preciso criar um novo espaço vetorial que represente todas as combinações dos qubits que o compõem. Para isso, a operação do produto tensorial, apresentada na \cref{sub:kronecker}, pode ser utilizada para fazer o produto do espaço vetorial de 2 qubits separados, criando um espaço vetorial que presente 2 qubits. Essa operação pode ser aplicada sucessivamente para todos os qubits do circuito.

Na notação bra-ket \cite{Dirac_1939}, múltiplos qubits podem ser representados por meio de um ket com o valor de todos os qubits. Por exemplo $\ket{0} \otimes \ket{0} = \ket{00}$. Então, um sistema com 2 qubits pode se encontrar numa superposição de 4 estados diferentes: \ket{00}, \ket{01}, \ket{10} e \ket{11}. Assim, nota-se que o número de possíveis estados cresce exponencialmente em relação ao número de qubits no circuito, visto que cada novo qubit duplica o número de possibilidades.

% TODO: Reescrever esse parágrado considerando que a CNOT já foi introduzida 
Dada a presença de múltiplos qubits no circuito, surgem, então, portas quânticas envolvam mais de um qubit. As portas de múltiplos qubits mais importantes são as portas controladas (outras portas normalmente podem ser decompostas em combinações de portas de um único qubit ou portas controladas). As portas controladas são portas que atuam sobre um qubit alvo, tendo como controle um grupo de vários outros qubits, fazendo efeito sobre o qubit alvo apenas no caso em que todos os qubit de controle estejam ativos. Por exemplo, considere o estado $\ket{\psi} = \frac{1}{\sqrt{2}} \cdot (\ket{00} + \ket{10})$. Aplicando uma porta X controlada (também conhecida como CNOT) no segundo qubit do circuito tendo o primeiro como controle, a porta irá agir apenas no caso \ket{10}, transformando-o em \ket{11}.

\begin{equation}
  CNOT \cdot \frac{1}{\sqrt{2}} \cdot (\ket{00} + \ket{10}) = \frac{1}{\sqrt{2}} \cdot (\ket{00} + \ket{11})
\end{equation}

O comportamento dessa porta pode ser descrito por meio da matriz apresentada na \cref{eq:cnot}

Note que a matriz possui dimensão 4x4, para se adequar ao formato do vetor que representa um circuito de 2 qubits (uma dimensão para cada combinação de estados). Assim como os vetores que representam a superposição dos estados, as matrizes que representam as portas do circuito crescem exponencialmente de acordo com o número de qubits envolvidos na porta.

\subsection{Produto Tensorial}\label{sub:kronecker}

O produto tensorial é uma operação essencial para a construção de matrizes e vetores que representem a interação entre os diferentes elementos presentes em um mesmo circuito quântico. Apesar de que portas quânticas normalmente operam sobre poucos qubits, para que uma matriz represente a porta para qubits específicos e seja compatível com a representação vetorial de múltiplos qubits, essa matriz precisa envolver a informação de que os demais qubits não serão afetados. Dessa forma, é como se todas as portas, na verdade, atuassem sobre todos os qubits do circuito e simplesmente o resultado de certos qubits é igual à entrada.

A operação consiste em estender o segundo operando seguindo o formato ditado pelo primeiro, da seguinte forma:
\begin{equation}
    \begin{array}{c}
      \left[ \begin{array}{c}
        1 \\
        0 \\
      \end{array} \right]
      \otimes
      \left[ \begin{array}{c}
        0 \\
        1 \\
      \end{array} \right]
    \end{array}
    =
    \begin{array}{c}
      \left[ \begin{array}{c}
       1
        \left[ \begin{array}{c}
        0 \\
        1 \\
      \end{array} \right] \\
        0
        \left[ \begin{array}{c}
        0 \\
        1 \\
      \end{array} \right] \\
      \end{array} \right]
    =
    \begin{array}{c}
      \left[ \begin{array}{c}
        0 \\
        1 \\
        0 \\
        0
      \end{array} \right]
    \end{array}
  \end{array}
\end{equation}

A mesma operação se estende para matrizes, permitindo que múltiplas portas sejam aplicadas sobre múltiplos qubits. Para isso, basta realizar o produto tensorial entre as matrizes de cada porta na ordem de seus qubits alvos. Por exemplo, na \cref{eq:kroneckerxh}, a matriz resultante representa a aplicação da porta X no primeiro qubit do circuito e da porta H no segundo.
\begin{equation}
  \begin{bmatrix}
    0 & 1 \\
    1 & 0
  \end{bmatrix}
  \otimes
  \begin{bmatrix}
    \frac{1}{\sqrt{2}} & \frac{1}{\sqrt{2}} \\
    \frac{1}{\sqrt{2}} & \frac{-1}{\sqrt{2}}
  \end{bmatrix}
  =
  \left[
  \begin{array}{cc}
  0\begin{bmatrix}
  \frac{1}{\sqrt{2}} & \frac{1}{\sqrt{2}} \\
  \frac{1}{\sqrt{2}} & \frac{-1}{\sqrt{2}}
  \end{bmatrix}
  &
  1\begin{bmatrix}
  \frac{1}{\sqrt{2}} & \frac{1}{\sqrt{2}} \\
  \frac{1}{\sqrt{2}} & \frac{-1}{\sqrt{2}}
  \end{bmatrix}
  \\[1ex]
  1\begin{bmatrix}
  \frac{1}{\sqrt{2}} & \frac{1}{\sqrt{2}} \\
  \frac{1}{\sqrt{2}} & \frac{-1}{\sqrt{2}}
  \end{bmatrix}
  &
  0\begin{bmatrix}
  \frac{1}{\sqrt{2}} & \frac{1}{\sqrt{2}} \\
  \frac{1}{\sqrt{2}} & \frac{-1}{\sqrt{2}}
  \end{bmatrix}
  \end{array}
  \right]
  =
  \begin{bmatrix}
    0 & 0 & \frac{1}{\sqrt{2}} & \frac{1}{\sqrt{2}} \\
    0 & 0 & \frac{1}{\sqrt{2}} & \frac{-1}{\sqrt{2}} \\
    \frac{1}{\sqrt{2}} & \frac{1}{\sqrt{2}} & 0 & 0 \\
    \frac{1}{\sqrt{2}} & \frac{-1}{\sqrt{2}} & 0 & 0
  \end{bmatrix}
  \label{eq:kroneckerxh}
\end{equation}

\section{Representação de Circuitos Quânticos}
Para facilitar a descrição e visualização de algoritmos quânticos, a notação de circuitos pode ser utilizada. Nela, cada qubit é representado por uma linha, a qual passará por caixas que representam as portas quânticas. Por exemplo, o circuito da \cref{fig:bellcircuit} começará no estado \ket{00} e será aplicada uma porta H no primeiro qubit e então uma CNOT no segundo qubit, tendo o primeiro como controle, gerando o estado $\frac{1}{\sqrt{2}} \cdot (\ket{00}+\ket{11})$.

\begin{figure}[h]
  \[
    \begin{quantikz}
      \ket{0} & \gate{H} & \ctrl{1} & \qw \\
      \ket{0} & \qw      & \targ{}  & \qw
    \end{quantikz}
  \]
  \caption{Circuito com múltiplos qubits}
  \label{fig:bellcircuit}
\end{figure}

\section{Representação do Circuito por Meio de Matrizes}\label{sec:matrixrep}

Como explicado na \cref{sub:kronecker}, todo circuito quântico pode ser representado como um vetor inicial que descreve a superposição de todos os qubits do circuito e as várias matrizes que representam cada uma das portas quânticas utilizadas no circuito. Após a conversão de todas as portas em matrizes, pode-se multiplicar todas elas para formar uma única matriz que represente o circuito inteiro. Também é possível simular a execução passo a passo, aplicando uma matriz de cada vez, de acordo com a ordem estabelecida pelo circuito, sobre o vetor de superposição do sistema \cite{nielsen2010quantum}.

Por exemplo, no circuito da \cref{fig:circuit3qubit}, a aplicação da porta X sobre o segundo qubit mais siginificativo pode ser representada por meio da matriz $I \otimes X \otimes I$, sendo I a matriz identidade. Ou seja, o primeiro e o último qubits não serão alterados, enquanto o segundo será negado.
\begin{figure}[h]
  \[
    \begin{quantikz}
      \ket{0} & \qw      & \qw & \ket{0} \\
      \ket{0} & \gate{X} & \qw & \ket{1} \\
      \ket{0} & \qw      & \qw & \ket{0}
    \end{quantikz}
    \qquad
    \Leftrightarrow
    \qquad
    I \otimes X \otimes I \cdot \begin{bmatrix}
      1 \\ 0 \\ 0 \\ 0 \\ 0 \\ 0 \\ 0 \\ 0
    \end{bmatrix}
    =
    \begin{bmatrix}
      0 \\ 0 \\ 1 \\ 0 \\ 0 \\ 0 \\ 0 \\ 0
    \end{bmatrix}
  \]
  \caption{Circuito com 3 qubits}
  \label{fig:circuit3qubit}
\end{figure}

Neste trabalho, a simulação de um circuito será feita por meio da conversão das portas lógicas em matrizes e a multiplicação uma a uma, de forma a emular cada passo de computação. O resultado, então, multiplica o vetor que representa o estado no qual todos os qubits estão em \ket{0}. Isso faz com que todo circuito seja inicializado em $\ket{0}^{\otimes n}$ para os n qubits do circuito. O vetor resultante pode então ser utilizado para extração de informações do resultado da simulação.



% ---

% ---
% 3 - Capítulo 3
% ---
\chapter{QMDD}\label{chap:qmdd}

{\itshape Quantum Multiple-Valued Decision Diagram} é uma estrutura de dados composta de um grafo dirigido acíclico cujos vértices são rotulados para representar um dos qubits do circuito. Por meio dessas informações, é possível percorrer o grafo de forma a reconstruir completamente a matriz que ele representa. Assim, o QMDD se mostra uma forma sem perdas de compactar matrizes. Além disso, é possível manipular QMDDs e realizar operações como adição e multiplicação entre eles sem precisar reconstruir suas matrizes, fazendo com que essas operações também tirem vantagem da capacidade de compressão oferecida pelo QMDD. \cite{miller2006qmdd}

Para que o diagrama não cresça exponencialmente em relação ao número de qubits no circuito, os algoritmos também devem se ater a preservar as seguintes restrições para o grafo do QMDD:

\begin{enumerate}
  \item Existe um único vértice que não possui arestas saintes. Esse vértice é chamado de vértice terminal.
  \item Todos os vértices que não são o vértice terminal possuem um rótulo indicando qual qubit o vértice representa e um conjunto de $r^2$ arestas saintes. \cite{fujita1997multi} No caso de QMDDs para circuitos quânticos, $r=2$, logo cada vértice possui 4 arestas.
  \item Existe uma única aresta sem vértice de origem. Essa é a aresta pela qual se iniciam todas as buscas pelo grafo e aponta para o vértice chamado de inicial.
  \item Todas as arestas do grafo são ponderadas por um valor complexo.
  \item Os rótulos dos vértices são ordenáveis, de forma que os rótulos de todo caminho formado do vértice inicial ao terminal estejam ordenados. Ou seja, em uma busca pelo gráfico, sempre se encontrará os vértices de cada qubit na mesma ordem. Além disso, essa ordem diz que o qubit menos significativo do circuito é o mais próximo do vértice terminal, enquanto o mais significativo é o próprio vértice inicial.
  \item Nenhum vértice não terminal é redundante. Ou seja, nenhum vértice possui todas as suas arestas com mesmo peso e apontando para um mesmo vértice. Caso todas as arestas tenham peso 0, o vértice ainda é considerado redundante mesmo que os vértices destino sejam diferentes. Isso porque a presença de um 0 no caminho entre o vértice inicial e final anula inteiramente o valor do caminho, fazendo com que qualquer aresta com peso 0 possa levar imediatamente ao vértice terminal sem alterar o valor do QMDD.
  \item Todo nodo não terminal é sempre normalizado, de forma que a primeira aresta de valor não nulo tem sempre valor 1. Note que deve sempre existir pelo menos uma aresta não nula, pois se todas as arestas tiverem peso 0, o vértice seria redundante.
  \item Todos os vértices são únicos. Ou seja, não existe um par de vértices com mesmo rótulo e todas as arestas iguais.
\end{enumerate}

Como estabelecido na \cref{sec:matrixrep}, para que um circuito quântico possa ser convertido em matrizes para ser simulado, são necessárias 3 operações fundamentais: construção da matriz a partir da porta; multiplicação de matrizes; medição. Nesta seção, serão apresentados os algoritmos necessários para realização dessas operações, em conjunto das subrotinas necessárias para os algoritmos. Por exemplo, a multiplicação de QMDDs faz uso da subrotina de adição de QMDDs.

Além disso, também será feita uma breve introdução ao conceito de grafos, e como eles oferecem formas eficazes de percorrer dados por meio de caminhos, uma funcionalidade essencial para a manipulação e recuperação das informações contidas em QMDDs.

\section{Como Ler um QMDD}\label{sec:qmddvis}

A representação gráfica de um QMDD permite a visualização de como um determinado valor da matriz codificada pode ser extraído por meio de um caminho entre o vértice raiz e o vértice terminal. A \Cref{fig:hqmdd} representa a matriz H. QMDDs que representam uma porta sobre um único qubit podem facilmente ser construídos, precisando apenas de um vértice inicial e um terminal, sendo as 4 arestas do vértice inicial os 4 valores presentes em cada posição da matriz. Note que o diagrama foi normalizado de acordo com o algoritmo de normalização apresentado em \cref{alg:normalization}, fazendo com que $\frac{1}{\sqrt{2}}$, o fator em comum a todas as arestas, seja fatorado e colocado na aresta raiz do diagrama.

\begin{figure}[h]
  \centering
  \begin{tikzpicture}[
    node distance=1.5cm,
    root/.style={circle,draw,minimum size=8mm},
    leaf/.style={rectangle,draw,minimum size=6mm},
    baseline={(current bounding box.center)}
  ]
    % terminal node
    \node[leaf] (t) at (0,0) {$1$};

    % nodes
    \node[root] (v) at (0,2) {q0};

    % incoming edge of weight 1
    \node[above=of v] (src) {};
    \draw[->] (src) -- node[right] {$\frac{1}{\sqrt{2}}$} (v);

    \draw[->] (v) edge[bend left=30]  node[right]  {$1$} (t);
    \draw[->] (v) edge[bend right=30] node[right] {$1$} (t);
    \draw[->] (v) edge[bend left=90]  node[right]  {$-1$} (t);
    \draw[->] (v) edge[bend right=90] node[right] {$1$} (t);
  \end{tikzpicture}
  \caption{QMDD representando a porta de Hadamard.}
  \label{fig:hqmdd}
\end{figure}

Já para matrizes que representam portas sobre múltiplos qubits, a ligação direta entre arestas e posições da matriz deixa de existir. Por exemplo, a matriz $X \otimes H$ pode ser representada com apenas 8 arestas, apesar de ser uma matriz com 16 posições, como demonstrado na \cref{fig:xhqmdd}.

\begin{figure}[h]
  \centering
  \begin{tikzpicture}[
    node distance=1.5cm,
    root/.style={circle,draw,minimum size=8mm},
    leaf/.style={rectangle,draw,minimum size=6mm},
    baseline={(current bounding box.center)}
  ]

    % terminal node
    \node[leaf] (t) at (0,0) {$1$};

    % nodes
    \node[root] (v) at (0,4) {q1};
    \node[root] (B) at (0,2) {q0};


    % incoming edge of weight 1
    \node[above=of v] (src) {};
    \draw[->] (src) -- node[right] {$\frac{1}{\sqrt{2}}$} (v);

    % four outgoing edges, one for each matrix entry (1,0,0,1)
    \draw[->] (v) edge[bend left=90]  node[right]  {$0$} (B);
    \draw[->] (v) edge[bend left=30]  node[right]  {$1$} (B);
    \draw[->] (v) edge[bend right=30] node[right] {$1$} (B);
    \draw[->] (v) edge[bend right=90] node[right] {$0$} (B);

    \draw[->] (B) edge[bend left=30]  node[right]  {$1$} (t);
    \draw[->] (B) edge[bend right=30] node[right] {$1$} (t);
    \draw[->] (B) edge[bend left=90]  node[right]  {$-1$} (t);
    \draw[->] (B) edge[bend right=90] node[right] {$1$} (t);


  \end{tikzpicture}
  \caption{QMDD representando uma porta X aplicada no primeiro qubit e uma porta H no segundo qubit de um circuito}
  \label{fig:xhqmdd}
\end{figure}

Para reconstruir a matriz representada por esse diagrama, basta seguir o caminho da aresta inicial ao vértice final. Cada vértice é rotula um qubit e suas arestas representam uma das 4 possíveis escolha para ler o qubit: começou em 0 e está em 0; começou em 0 e está em 1; começou em 1 e está em 0 ou começou em 1 e está em 1. Como os qubits normalmente são inicializados em \ket{0}, pode-se construir o caminho considerando apenas a primeira e a segunda aresta do vértice. Caso o qubit do vértice esteja em \ket{0} no estado a ser lido, segue a primeira aresta e caso esteja em \ket{1} segue a segue.

Uma forma mais algorítmica para recuperar um determinado valor da matriz é construir o caminho a partir dos índices do valor na matriz. Por exemplo, para obter o valor $M_{31}$ (valor na quarta linha e segunda coluna da matriz), é preciso escrever o índice em binário utilizando o número de bits igual ao número de qubits indicado pelo diagrama, no caso (3, 1) vira (11, 01). Então, para escolher a aresta do i-ésimo vértice, monta-se o par linha, coluna do i-ésimo bit mais significativo do índice. Ou seja, no qubit 0, a aresta escolhida será terceira (índice $2_{10}$, ou $10_2$) e a aresta do qubit 1 será a quarta (índice $3_{10}$, ou $11_2$).

Ao atingir o vértice terminal, basta multiplicar os pesos de todas as arestas presentes no caminho tomado para recuperar o valor representado pela matriz original.

\begin{equation}
  M_{31} = \frac{1}{\sqrt{2}} \cdot 1 \cdot -1 = -\frac{1}{\sqrt{2}}
\end{equation}

Note que, na \cref{fig:xhqmdd}, as arestas de peso 0 levam ao vértice q1 para facilitar a visualização do grafo. Uma otimização importante é que toda aresta de peso 0 pode levar diretamente ao vértice terminal, já que a presença de um 0 no caminho faz com que o resultado seja 0 independentemente de quais outros valores estiverem presentes.

\section{Grafos}\label{sec:graph}

Visto que QMDDs dependem integralmente de grafos, esta seção busca introduzir os conceitos necessários para compreender as definições do QMDD e as restrições necessárias para que um grafo seja válido para um QMDD \cite{jungnickel}

Grafo é o nome dado para estruturas de dados organizadas em elementos e ligações entre esses elementos. Cada elemento do grafo é chamado de vértice e as ligações entre os elementos são chamadas de arestas. Um cenário no qual grafos são úteis, por exemplo, é a representação entre diferentes pontos de ônibus de uma cidade e a existência de uma rota que ligue 2 pontos. Nesse caso, cada ponto de ônibus é um vértice e sempre que exista uma linha de ônibus levando de um ponto A para um ponto B, vai existir uma aresta ligando os vértices A e B. Arestas podem ser representadas como uma tupla com o vértice de saída e o vértice destino da aresta.

O grafo da \cref{fig:graph2nodes} é composto do conjunto de vértices \{A, B\} e do conjunto de arestas \{(A, B)\}.

\begin{figure}[h]
  \centering
  \begin{tikzpicture}[>=stealth, node distance=2cm]

  \node (A) [circle, draw] {A};
  \node (B) [circle, draw, right of=A] {B};

  \draw[->] (A) -- (B);

  \end{tikzpicture}
  \caption{Grafo com 2 vértices}
  \label{fig:graph2nodes}
\end{figure}


\subsection{Caminhos}
A existência de uma aresta saindo de A e levando a B significa que existe um caminho de A para B. Nem todos os vértices do precisam estar diretamente ligados por uma aresta, é possível que exista um caminho indireto entre dois vértices. Por exemplo, no grafo da imagem \cref{fig:graph3node}, existe um caminho de A para B e um de B para C, isso significa que, indiretamente, também existe um caminho de A para C \cite{jungnickel}.

\begin{figure}[h]
  \centering
  \begin{tikzpicture}[>=stealth, node distance=2cm]

  \node (A) [circle, draw] {A};
  \node (B) [circle, draw, right of=A] {B};
  \node (C) [circle, draw, right of=B] {C};

  \draw[->] (A) -- (B);
  \draw[->] (B) -- (C);

  \end{tikzpicture}
  \label{fig:graph3node}
\end{figure}

\subsection{Arestas Ponderadas}
Além de armazenar apenas a informação de que se existe ou não um caminho entre dois vértices, uma aresta também pode armazenar um valor para esse caminho. Normalmente esse valor é entendido como o custo da aresta, porém, nos QMDDs, esse valor é interpretado como um dos valores que compõem a matriz representada.

No exemplo dos pontos de ônibus, o valor de uma aresta poderia representar o tempo que leva para chegar de um lugar a outro. Isso se torna útil em junção com o conceito de caminho indireto: se o caminho de A para C é na verdade o caminho de A para B seguido do caminho de B para C, sabe-se que o custo para chegar em C a partir de A será a soma dos custos para ir de A a B e de B a C. Por exemplo, se a aresta (A, B) tem valor 10 minutos e a aresta (B, C) tem valor 20 minutos, sabe-se que é possível sair de A e chegar em C em 30 minutos.

\begin{figure}[h]
  \centering
  \begin{tikzpicture}[>=stealth, node distance=2cm]

  \node (A) [circle, draw] {A};
  \node (B) [circle, draw, right of=A] {B};
  \node (C) [circle, draw, right of=B] {C};

  \draw[->] (A) -- (B) node[midway, above] {10};
  \draw[->] (B) -- (C) node[midway, above] {20};

  \end{tikzpicture}
  \label{fig:graph3nodeweighted}
\end{figure}


\subsection{Grafos Dirigidos e Ciclos}
Note que a possibilidade de ir do ponto A para o ponto B não implica na existência de um caminho de B para A. Por isso a ordem da tupla é importante para as arestas. Dessa forma, grafos podem ou não ser dirigidos, ou seja, é possível impor a restrição de que toda aresta representa a existência de ambos os caminhos, tornando a ordem da tupla irrelevante. Caso o grafo não seja dirigido, toda aresta resulta na existência de um ciclo, ou seja, existe um caminho de um vértice para ele mesmo. No caso da aresta não dirigida (A, B), existe um caminho de A para B e existe um caminho de B para A, logo existe um caminho de A para A.

Por outro lado, no caso de grafos dirigidos, a existência de um ciclo passa a ser apenas uma possibilidade facilmente identificada. No caso dos QMDDs, essa restrição é sempre respeitada pela forma como os algoritmos de construção e manipulação são definidos. Porém, dado um grafo arbitrário, é possivel verificar se existe um ciclo por meio de uma busca a partir de cada vértice, verificando se existe um caminho que leve de volta ao mesmo vértice \cite{jungnickel}.

O presente grafo na \cref{fig:acyclicgraph} é acíclico, visto que não existe caminho de nenhum vértice para si mesmo, já o grafo na \cref{fig:cyclicgraph}, existe um caminho de A para A, passando pelos vértices B e C, logo esse grafo é cíclico.

\begin{figure}[h]
  \centering
  \begin{minipage}{0.45\textwidth}
    \centering
    \begin{tikzpicture}[>=stealth, every node/.style={circle,draw,minimum size=7mm}]
      \node (A) at (90:2)  {A};
      \node (B) at (210:2) {B};
      \node (C) at (330:2) {C};
      \draw[->] (A) -- (B);
      \draw[->] (B) -- (C);
      \draw[->] (A) -- (C);
    \end{tikzpicture}
    \caption{Grafo acíclico}
    \label{fig:acyclicgraph}
  \end{minipage}
  \qquad
  \begin{minipage}{0.45\textwidth}
    \centering
    \begin{tikzpicture}[>=stealth, every node/.style={circle,draw,minimum size=7mm}]
      \node (A) at (90:2)  {A};
      \node (B) at (210:2) {B};
      \node (C) at (330:2) {C};
      \draw[->] (A) -- (B);
      \draw[->] (B) -- (C);
      \draw[->] (C) -- (A);
    \end{tikzpicture}
    \caption{Grafo cíclico}
    \label{fig:cyclicgraph}
  \end{minipage}
\end{figure}

\subsection{Representação de Vértices e Arestas}

No começo desta seção, vértices foram apresentadas como sendo elementos arbitrários de um conjunto e arestas como sendo tuplas dos vértices conectados e, opcionalmente, um peso. Para a descrição dos algoritmos do QMDD, uma notação diferente será utilizada. Nela, cada aresta possui apenas um valor e um vértice destino, enquanto cada vértice será uma tupla com seu rótulo e um conjunto de arestas.

Assim, o grafo da \cref{fig:graph3nodeweighted} pode ser representado como o conjunto de vértices $\{(A, \{(10, B)\}), (B, \{(20, C)\}), (C, \{\})\}$

\section{Algoritmos}

A seguir, serão definidas operações entre arestas de QMDDs. Dessa forma, a operação entre dois QMDDs é realizada por meio da operação entre suas duas arestas iniciais.

Cada operação é definida recursivamente, finalizando ao encontrar arestas terminais. Assim, cada operação retorna um novo QMDD, começando a construção pelo vértice terminal e adicionando os novos vértices e arestas sucessivamente até completar a construção da aresta inicial.

\subsection{Funções Fundamentais}\label{sub:fundamentals}

Para definição sucinta dos algoritmos, serão utilizadas algumas notações para se referir a informações presentes nos vértices e arestas do QMDD.

\begin{enumerate}
  \item $x(v)$: $v$ é um vértice e $x(v)$ signfica o rótulo do vértice $v$;
  \item $v(e)$: $e$ é uma aresta e $v(e)$ representa o vértice ao qual a aresta aponta;
  \item $w(e)$: $e$ é uma aresta e $w(e)$ é o peso dessa aresta;
  \item $E_i(x)$: Quando $x$ é um vértice, $E_i(x)$ é a i-ésima aresta do vértice. Quando $x$ é uma aresta, $E_i(x)$ é a i-ésima aresta do vértice ao qual $x$ aponta;
  \item $T(e)$: $e$ é uma aresta e $T(e)$ é um valor verdadeiro ou falso que indica se a aresta é terminal. Ou seja, se o vértice ao qual $e$ aponta é o vértice terminal;
  \item $v_t$: é o vértice terminal do QMDD, que não contém nenhuma aresta.

\end{enumerate}


Além disso, note que, para representar as matrizes utilizadas por circuitos quânticos, o número de arestas por vértice do QMDD será sempre 4. Apesar de ser possível que diferentes usos de matrizes requeiram diferentes números de arestas por vértice, os algoritmos serão descritos com os valores fixos para as matrizes relevantes para a simulação de circuitos quânticos.

\subsection{Normalização e Eliminação de Redundância}

Normalizar um vértice é essencial para possibilitar a detecção de redundância presente em um QMDD, visto que a presença de um múltiplo em comum a todas as arestas de um mesmo vértice pode ser fatorado e adicionado à aresta que aponta para esse vértice. Dessa forma, a equivalência entre diferentes vértices pode ser encontrada apenas comparando o alvo e peso de suas arestas.

O processo de normalizar um vértice também envolve a remoção de vértices reduntantes. Ou seja, vértices cujas arestas sejam todas iguais entre si podem ser substituídos por um caminho direto para o alvo de suas arestas. Assim, o algoritmo de normalização deve retornar uma aresta que aponta ou para o vértice original (caso o vértice não seja redundante) ou para o próximo vértice (caso o original seja reduntante). Por isso, é importante que os algoritmos sejam definidos sobre as arestas e não sobre os vértices. Assim, para normalizar um vértice, o algoritmo deve ser aplicado sobre a aresta que aponta para esse vértice.

Além disso, como as outras operações constroem seu resultado do vértice final ao inicial, o fato que o algoritmo de normalização retorna a aresta que aponta para o vértice faz com que o efeito da normalização seja propagado do vértice final ao inicial. 

% TODO: Mover este exemplo para o algoritmo de adição

% Garantir que os vértices estão normalizados é um esforço adicional para cada um dos algoritmos, visto que, por exemplo, a soma de dois vértices normalizados pode resultar em um vértice não normalizado. Imagine dois vértices, um com arestas de peso 1, 0, 0 e 1 e outro vértice com arestas de peso 0, 1, 1 e 0, todas apontadas para o mesmo vértice. Apesar de estarem normalizados, a soma dos dois vértices resultará em um vértice com arestas de peso 1, 1, 1 e 1, todas apontando para o mesmo vértice. Esse vértice por sua vez é reduntante, já que todas as arestas possuem mesmo peso e apontam para o mesmo vértice. Logo, é necessário normalizá-lo antes de retornar o resultado da soma.


% \[
%   \begin{tikzpicture}[
%     node distance=1.5cm,
%     root/.style={circle,draw,minimum size=8mm},
%     leaf/.style={rectangle,draw,minimum size=6mm},
%     baseline={(current bounding box.center)}
%   ]

%     % terminal node
%     \node[leaf] (t) at (0,0) {$1$};

%     % QMDD internal node
%     \node[root] (v) at (0,2) {};

%     % incoming edge of weight 1
%     \node[above=of v] (src) {};
%     \draw[->] (src) -- node[right] {$1$} (v);

%     % four outgoing edges, one for each matrix entry (1,0,0,1)
%     \draw[->] (v) edge[bend left=90]  node[right]  {$1$} (t);
%     \draw[->] (v) edge[bend left=30]  node[right]  {$0$} (t);
%     \draw[->] (v) edge[bend right=30] node[right] {$0$} (t);
%     \draw[->] (v) edge[bend right=90] node[right] {$1$} (t);

%   \end{tikzpicture}
%   \;+\;
%   \begin{tikzpicture}[
%     node distance=1.5cm,
%     root/.style={circle,draw,minimum size=8mm},
%     leaf/.style={rectangle,draw,minimum size=6mm},
%     baseline={(current bounding box.center)}
%   ]

%     \node[leaf] (t) at (0,0) {$1$};
%     \node[root] (v) at (0,2) {};

%     \node[above=of v] (src) {};
%     \draw[->] (src) -- node[right] {$1$} (v);

%     \draw[->] (v) edge[bend left=90]  node[right]  {$0$} (t);
%     \draw[->] (v) edge[bend left=30]  node[right]  {$1$} (t);
%     \draw[->] (v) edge[bend right=30] node[right] {$1$} (t);
%     \draw[->] (v) edge[bend right=90] node[right] {$0$} (t);

%   \end{tikzpicture}
%   \;=\;
%   \begin{tikzpicture}[
%     node distance=1.5cm,
%     root/.style={circle,draw,minimum size=8mm},
%     leaf/.style={rectangle,draw,minimum size=6mm},
%     baseline={(current bounding box.center)}
%   ]

%     \node[leaf] (t) at (0,0) {$1$};
%     \node[root] (v) at (0,2) {};

%     \node[above=of v] (src) {};
%     \draw[->] (src) -- node[right] {$1$} (v);

%     \draw[->] (v) edge[bend left=90]  node[right]  {$1$} (t); % (0,0)
%     \draw[->] (v) edge[bend left=30]  node[right]  {$1$} (t); % (0,1)
%     \draw[->] (v) edge[bend right=30] node[right] {$1$} (t); % (1,0)
%     \draw[->] (v) edge[bend right=90] node[right] {$1$} (t); % (1,1)

%   \end{tikzpicture}
%   \quad\Longrightarrow\quad
%   \begin{tikzpicture}[
%     node distance=1.5cm,
%     root/.style={circle,draw,minimum size=8mm},
%     leaf/.style={rectangle,draw,minimum size=6mm},
%     baseline={(current bounding box.center)}
%   ]

%     % terminal node
%     \node[leaf] (t) at (0,0) {$1$};

%     % incoming edge of weight 1
%     \node[above=of t] (src) {};
%     \draw[->] (src) -- node[right] {$1$} (t);

%   \end{tikzpicture}
% \]

O algoritmo em si se divide em duas etapas: primeiro verifica-se se o vértice é redundante e, se não for, normaliza-se os pesos de suas arestas.

Para identificar a redundância do vértice, basta comparar todas as suas arestas. Se todas forem iguais (mesmo peso e mesmo vértice alvo), o vértice pode ser descartado, então basta retornar uma aresta com o peso e alvo das arestas do vértice.

Caso o vértice não seja redundante, o peso de todas as suas arestas será dividido pelo peso da primeira aresta com peso diferente de 0. Isso faz com que todo vértice tenha sua primeira aresta não nula com peso 1. Então, basta retornar uma nova aresta, apontando para o mesmo vértice e com peso sendo o peso original multiplicado pelo fator em comum às arestas do vértice \cite{miller2006qmdd}.

\begin{algorithm}[H]
  \caption{Normalização de Vértice da Aresta}
  \label{alg:normalization}
  \KwIn{Aresta $e$}
  \KwOut{Aresta}

  \If{$E_0(e) = E_1(e) = E_2(e) = E_3(e)$}{
    \Return $E_0(e)$
  }
  \Else{
  % TODO: Isso não tá ignorando as arestas nulas.
    $w \gets w(E_0(e))$ \;
    \For{$i \gets 0$ \KwTo $3$}{
      $w(E_i(e)) \gets w(E_i(e))/w$
    }
  }

  $w(e) \gets w * w(e) $

  \Return $e$

\end{algorithm}

A seguir, são oferecidos dois exemplos de normalização de vértices. No primeiro, o vértice é reduntante pois todas suas arestas são iguais, logo é descartado. Já no segundo, a aresta nula é ignorada enquanto as demais tem seus pesos dividos pelo peso da primeira aresta não nula.

% TODO: vértice 1 1 1 1 -> ø; vértice 0 2 1 1 -> 0 1 0.5 0.5

\subsection{Adição}

Apesar de não ser uma operação diretamente necessária para simulação de circuitos, a adição de matrizes é um passo essencial para a multiplicação de matrizes, que, por sua vez, é necessária para a simulação.

% TODO: Reescrever a última frase
Para realizar a adição de duas arestas, cria-se um novo vértice, o qual armazenará o resultado da soma das duas arestas e, então, uma nova aresta é retornada, apontando para o vértice criado. As arestas desse vértice serão as somas, par a par, das arestas dos vértices a serem somados, multiplicadas pelos pesos das arestas de entrada. Ou seja, a primeira aresta do novo vértice será a soma da primeira aresta do primeiro vértice com a primeira aresta do segundo vértice.

Caso pelo menos uma das arestas seja terminal, a construção da soma pode começar, visto que, como a aresta terminal não terá um vértice com arestas para serem somadas, basta somar o peso da aresta terminal ao peso da outra aresta. A seguir, na descrição do algoritmo, esse passo é otimizado, de forma a evitar o cálculo desnecessário caso uma das arestas tenha peso 0. Assim, se uma das arestas tiver peso 0, basta retornar o valor da outra aresta, evitando o recálculo do valor da nova aresta \cite{miller2006qmdd}.

\begin{algorithm}[H]
  \caption{Adição de Arestas}
  \label{alg:addition}
  \KwIn{Aresta $e_0$, Aresta $e_1$}
  \KwOut{Aresta}

  \If{$T(e_1)$}{
    $e_0, e_1 \gets e_1, e_0$
  }
  \If{$T(e_0)$}{
    \If{$w(e_0) = 0$}{
      \Return $e_1$
    }
    \If{$T(e_0)$}{
      $e_n \gets Aresta(w(e_0)+w(e_1), v(e_0))$ \;
      \Return $e_n$
    }
  }
  $v_n \gets Vertice(x(e_0), \emptyset)$ \;
  \For{$i \gets 0$ \KwTo $3$}{
    $p \gets Aresta(w(e_0)*w(E_i(e_0)), v(E_i(e_0)))$ \;
    $q \gets Aresta(w(e_1)*w(E_i(e_1)), v(E_i(e_1)))$ \;
    $E_i(v_n) \gets p + q$
  }
  \Return $normalize(Aresta(1, v_n))$

\end{algorithm}

\subsection{Multiplicação} \cite{kydros2025tutorial}

O algoritmo de multiplicação de QMDDs segue a mesma lógica da multiplicação de matrizes. São construídas n matrizes cujos valores são a multiplicação da i-ésima posição da linha da primeira e a i-ésima posição da coluna da segunda. Essas matrizes são então somadas para obter o resultado da multiplicação.

% TODO: GRAWWWAAAAAAAAA
Na multiplicação de QMDDs, cada par de vértices cria um novo vértice cujas arestas são a soma dos n pares linha-coluna de suas arestas. Note que, como o vértice armazena suas arestas linearmente, deve-se interpretar que a primeira e segunda aresta compõem a primeira coluna da matriz, enquanto a terceira e quarta compõem a segunda coluna.

Ao atingir uma aresta final, basta multiplicar o peso dessa aresta pelo peso da outra e retorná-la. Assim como foi feito na adição, os casos em que o peso da aresta final é 0 ou 1 podem ser otimizados. No caso da aresta de peso 0, basta retornar a aresta de peso 0. Já no caso da aresta de peso 1, basta retornar a outra aresta.

\begin{algorithm}[H]
  \caption{Multiplicação de Arestas}
  \label{alg:multiplication}
  \KwIn{Aresta $e_0$, Aresta $e_1$}
  \KwOut{Aresta}

  \If{$T(e_1)$}{
    $e_0, e_1 \gets e_1, e_0$
  }
  \If{$T(e_0)$}{
    \If{$w(e_0) = 0 $}{
      \Return $e_0$
    }
    \If{$w(e_0) = 1$}{
      \Return $e_1$
    }
    \Return $Aresta(w(e_0)*w(e_1), v(e_1))$
  }
  $v_n \gets Vertice(x(e_0), \emptyset)$
  \For{$i \gets 0$ \KwTo $1$}{
    \For{$j \gets 0$ \KwTo $1$}{
      $E_{2*i+j}(v_n) \gets Aresta(0, v_t)$ \;
      \For{$k \gets 0$ \KwTo $1$}{
        $p \gets Aresta(w(e_0)*w(E_{2*i+k(e_0)}), v(E_{2*i+k}(e_0)))$ \;
        $q \gets Aresta(w(e_1)*w(E_{j+2*k(e_0)}), v(E_{j+2*k}(e_1)))$ \;
        $E_{2*i+j}(v_n) \gets E_{2*i+j}(v_n) + p*q$
      }
    }
  }
  $e_n \gets normalize(Aresta(1, v_n$)) \;
  \Return $e_n$

\end{algorithm}


\subsection{Construção a Partir de Portas}\label{sec:gateconstruct}

A construção de um QMDD a partir de uma porta quântica pode ser feita de duas formas: uma porta arbitrária para um único qubit ou uma porta arbitrária controlada, com um qubit alvo e múltiplos qubits de controle.

A construção de uma porta arbitrária se dá pelo produto tensorial entre matrizes identidade e a matriz que representa a porta arbitrária, de forma a produzir uma única matriz que opere sobre todos os qubits, mantendo todos exceto o qubit alvo da porta inalterados e aplicando o efeito da porta sobre o qubit desejado. Para isso, basta utilizar o algoritmo de produto tensorial definido na \cref{subsub:kronecker}. Assim, para aplicar uma porta U no i-ésimo qubit de um circuito com k qubits, basta realizar o produto tensorial $I^{\otimes i-1} \otimes U \otimes I^{\otimes k-i}$. 

Já a construção de portas controlodas possui duas etapas: uma para os qubits de controle mais significativos que o qubit alvo e uma para os menos significativos. Essas duas etapas são descritas na \cref{subsub:controlgate} \cite{miller2006qmdd, kydros2025tutorial}

\subsubsection{Produto Tensorial}\label{subsub:kronecker}

O produto tensorial consiste basicamente do encadeamento dos grafos de cada diagrama. $A \otimes B$ significa que as arestas terminais de A serão conectadas ao vértice inicial de B. Note que, por não criar novos vértices, o resultado do produto tensorial não consome mais memória que os dois valores de entrada do produto. Isso representa um ganho exponencial em relação à representação matricial do produto tensorial, cujo tamanho do resultado é o produto dos tamanhos das entradas.

Diferentemente das operações previamente mencionadas, o produto tensorial não avança sobre as duas arestas. Em vez disso, a segunda aresta da operação é levada adiante, de forma que apenas a primeira será alterada. Isso porque a parte mais relevante do produto tensorial acontece entre a aresta inicial do segundo QMDD e todas as arestas finais do primeiro QMDD. Assim, para realizar o produto tensorial, basta percorrer o primeiro QMDD até encontrar suas arestas terminais. Em cada aresta será aplicado o passo de ligar a aresta terminal do primeiro QMDD ao vértice inicial do segundo e a multiplicação dos pesos da aresta final com a aresta inicial.

Além disso, a forma como a navegação do primeiro QMDD é feita faz com que o fator da aresta inicial do segundo QMDD seja propagado até chegar na aresta inicial do primeiro QMDD, por meio do processo sucessivo de normalização das arestas.

\begin{algorithm}[H]
  \caption{Produto Tensorial Entre Arestas}
  \label{alg:kronecker}
  \KwIn{Aresta $e_0$, Aresta $e_1$}
  \KwOut{Aresta}
  
  \If{$T(e_0)$}{
    \If{$w(e_0) = 0$}{
      \Return $e_0$
    }
    \If{$w(e_0) = 1$}{
      \Return $e_1$
    }
    \Return $Aresta(w(e_0)*w(e_1), v(e_1))$
  }

  $v_n \gets Vertice(x(e_0), \emptyset)$ \;

  \For{$i \gets 0$ \KwTo $3$}{
    $E_i(v_n) \gets tensorial(E_i(e_0), e_1)$
  }
  \Return $normalize(Aresta(w(e_0), v_n))$
\end{algorithm}

\subsubsection{Construção de Portas Controladas}\label{subsub:controlgate}

Como dito na \cref{sec:gateconstruct}, a construção de portas controlodas se divide em duas partes: qubits de controle menos significativos que o alvo e qubits mais significativos. A construção do QMDD por meio desse método ocorre do vértice terminal ao vértice inicial, considerando apenas os qubits de controle e o qubit alvo. Para cada qubit de controle, é construído um QMDD que representa o comportamento de apenas propagar a transformação sobre o qubit alvo pelas arestas que representam o caso em que o qubit está ativo.

\begin{algorithm}[H]
  \caption{Construção de Portas Controlodas}
  \label{alg:controlgateconstruct}
  \KwIn{Índice do qubit alvo $t$; Conjunto de índices dos qubits de controle $C$; Matriz da porta $m$}
  \KwOut{Aresta}

  $r \gets \{Aresta(v_t, m00), Aresta(v_t, m01), Aresta(v_t, m10), Aresta(v_t, m11)\}$
  $x \gets {Aresta(0, v_t), Aresta(0, v_t), Aresta(0, v_t), Aresta(0, v_t)} $

  \For{$c \in C \land c < t$}{
    \For{$i \gets \{0, 1\}$}{
      \For{$j \gets \{0, 1\}$}{
        $k \gets 2*i + j$ \;
        $x_3 \gets r_k$ \;
        $x_0 \gets Aresta(1, v_t) if i = j, else Aresta(0, v_t)$ \;
        $r_k \gets normalize(Aresta(1, Vertice(c, x)))$
      }
    }
  }
  $e \gets normalize(Aresta(1, Vertice(t, r)))$
  \For{$c \in C \land c > t$}{
    $x_3 \gets e$ \;
    $x_0 \gets Aresta(1, T)$ \;
    $e \gets normalize(Aresta(1, Vertice(c, x)))$
  }
  \Return $e$
\end{algorithm}

\subsection{Medição}\label{sub:measurement}

A operação de medição sobre o estado também pode ser representada por meio de um produto matricial. Para construir a matriz que representa essa operação, basta aplicar as matrizes $M_0$ e $M_1$, apresentadas na \cref{eq:measurementmatrices} sobre o qubit a ser medido, de acordo com o resultado obtido. Assim, pode-se tirar vantagem dos algoritmos já definidos para multiplicação de matrizes e produto tensorial para construir a matrix $I^{\otimes n-1} \otimes (M_0 ou M_1) \otimes I^{\otimes k-n}$, sendo $k$ o número de qubits no circuito e $n$ o índice do qubit medido.

\begin{equation}
  % \caption{Matrizes de Medição}
  \label{eq:measurementmatrices}
  M_0
  =
  \begin{bmatrix}
    1 & 0 \\
    0 & 0
  \end{bmatrix}
  \qquad
  M_1
  =
  \begin{bmatrix}
    0 & 0 \\
    0 & 1
  \end{bmatrix}
\end{equation}

Além disso, para extrair um resultado que obedeça à distribuição de probabilidade ao final da execução do algoritmo, é possível percorrer o grafo do vértice inicial ao final escolhendo as arestas aleatoriamente de acordo com seus pesos. Para isso, é preciso considerar apenas duas das quatro arestas dos vértices, sendo as arestas 1 e 3 responsáveis pelo resultado caso o qubit do vértice tenha sido inicializado em \ket{0} e as arestas 2 e 4 para \ket{1}. \cite{kydros2025tutorial}

Esse método de obtenção de um único resultado aleatório de acordo com a distribuição probabilística do estado é chamada de simulação fraca \cite{wang2015weak}.

\section{Extração de Resultados}

QMMDs permitem a realização de simulação quântica forte. Isso significa que é possível extrair a probabilidade de cada resultado após a execução do circuito. Essa operação inevitavelmente possui custo exponencial, pois n qubits geram $2^n$ possíveis resultados, por isso estratégias de simulação fraca como a apresentada na \cref{sub:measurement} são utilizadas para circuitos com mais qubits \cite{wang2015weak}.

Caso o circuito possua uma quantia moderada de qubits, é possível realizar um procedimento semelhante à busca descrita na \cref{sub:measurement}. A diferença é que, ao invés de escolher arestas aleatoriamente, cada escolha leva ao valor que representa um resultado possível, logo basta visitar todos os caminhos possíveis.

Lembrando que o QMDD resultante do produto de todas as portas lógicas representa toda a transformação linear do circuito, envolvendo informações de resultados para circuitos inicializados em estados arbitrários e não apenas $\ket{0}^{\otimes n}$. Para extrair o resultado considerando que o circuito foi inicializado em $\ket{0}^{\otimes n}$, basta ignorar as arestas 2 e 4 de cada vértice.

% ---

% ---
% 4 - Conclusão
% ---
%\phantompart
\chapter{Planejamento para Relatório II}\label{chap:tcc2}

Na segunda etapa deste trabalho, os algoritmos descritos em \cref{chap:qmdd} serão implementados na linguagem de programação Rust, de forma a serem eventualmente integrados à plataforma de desenvolvimento quântico Ket.

Além disso, a implementação será utilizada para realizar experimentos de forma a documentar o desempenho do método de simulação apresentado e identificar a natureza dos cenários nos quais o método apresenta desempenho superior aos métodos convencionais.

\section{Cronograma}

\subsection{Implementação dos Algoritmos}

A primeira atividade a ser desenvolvida será a codificação dos algoritmos na linguagem de programação Rust em conjunto das estruturas de dados necessárias para execução eficaz das funções fundamentais descritas em \cref{sub:fundamentals}. 

\subsection{Validação da Implementação}

Em conjunto com a implementação, casos de teste serão elaborados, utilizando os simuladores já existentes no Ket para validar os resultados produzidos pela implementação Rust do QMDD.

\subsection{Documentação da Implementação e Integração com o Ket}

Para que o simulador seja apropriadamente adicionado como uma funcionalidade Ket, é necessário que ele cumpra com as interfaces definidas pela plataforma sejam satisfeitas. Nesta atividade, a forma como o simulador cumpre esses requisitos será descrita de forma a permitir futura compreensão do sistema por terceiros.

\subsection{Análise dos Experimentos e Conclusão Sobre Vantagem do QMDD}

Uma vez que o simulador esteja disponível para uso integrado com o Ket, diferentes algoritmos quânticos serão executados utilizando as várias opções de simulação disponíveis no Ket. Estatítistcas de desempenho de programas como tempo de execução e consumo de memória serão coletadas e comparadas entre os métodos para cada algoritmo.
% ---

% ----------------------------------------------------------
% ELEMENTOS PÓS-TEXTUAIS
% ----------------------------------------------------------
\postextual
% ----------------------------------------------------------

% ----------------------------------------------------------
% Referências bibliográficas
% ----------------------------------------------------------
\begingroup
    \SingleSpacing\printbibliography[title=REFERÊNCIAS]
\endgroup

% ----------------------------------------------------------
% Glossário
% ----------------------------------------------------------
%
% Consulte o manual da classe abntex2 para orientações sobre o glossário.
%
%\glossary

% ----------------------------------------------------------
% Apêndices
% ----------------------------------------------------------

% ---
% Inicia os apêndices
% ---
\begin{apendicesenv}
%	\partapendices* 
	\input{aftertext/apendice_a}
\end{apendicesenv}
% ---


% ----------------------------------------------------------
% Anexos
% ----------------------------------------------------------

% ---
% Inicia os anexos
% ---
\begin{anexosenv}
%	\partanexos*
	\input{aftertext/anexo_a}
\end{anexosenv}

%---------------------------------------------------------------------
% INDICE REMISSIVO
%---------------------------------------------------------------------
%\phantompart
%\printindex
%---------------------------------------------------------------------

\end{document}
