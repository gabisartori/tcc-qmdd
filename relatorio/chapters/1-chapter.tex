\chapter{Introdução}\label{chap:intro}

Conforme os fenômenos da mecânica quântica foram melhor descritos e compreendidos, um novo paradigma para solução de problemas surge: a computação quântica. Essa nova forma de elaborar algoritmos é equivalente à computação clássica em termos de computabilidade \cite{shor1998quantum}, porém apresenta vantagem em termos de complexidade computacional para certos problemas. Por exemplo, dada uma lista de items sem organização prévia, na computação clássica, a melhor forma de verificar se determinado item está presente nessa lista é percorrendo-a, elemento por elemento, buscando pelo alvo. Essa solução possui complexidade temporal O(n), o que significa que o tempo para encontrar um elemento específico cresce linearmente em relação ao tamanho da lista (uma lista 4 vezes maior que outra leva 4 vezes mais tempo para ser percorrida). Porém, Grover demonstrou que, usando da mecânica quântica, é possível resolver esse mesmo problema com uma complexidade temporal O($\sqrt{n}$) \cite{grover}. Nesse caso, uma lista 4 vezes maior que outra leva apenas o dobro do tempo para que o elemento desejado seja encontrado. Assim, sabe-se que é possível que algoritmos quânticos superem algoritmos clássicos em desempenho, nesse caso, como a complexidade foi reduizada de $O(n)$ para $O(\sqrt{n})$, diz-se que houve um ganho quadrático de desempenho.

Atualmente, porém, essa área de pesquisa encontra um obstáculo: a dificuldade de executar os algoritmos elaborados, afetando significativamente o desenvolvimento de novos algoritmos, já que testá-los torna-se uma tarefa manual e demorada sem um computador para realizá-la. Essa dificuldade existe porque hardware que possibilite a execução das operações fundamentais da computação quântica é caro e dificilmente acessível. Além disso, mesmo quando acessível, hardware quântico moderno apresenta baixa capacidade, permitindo apenas a execução de algoritmos simples sobre exemplos de casos pequenos \cite{golec2024quantum}.

Dessa forma, para auxiliar no desenvolvimento de algoritmos quânticos, busca-se solução na simulação por meio de computadores clássicos para compensar a falta de hardware quântico. Visto que ambas as formas de definir algoritmos são computacionalmente equivalentes, sabe-se que todo algoritmo quântico pode ser executado em uma máquina clássica e vice-versa. Então, basta definir uma forma de converter algoritmos quânticos em algoritmos clássicos e executar essa nova versão em uma máquina clássica, os quais podem ser executados em máquinas mais acessíveis e baratas. Neste trabalho, essa conversão será feita decompondo o circuito em matrizes, cada porta lógica é representada por uma matriz e a execução do algoritmo se dá pelo produto das matrizes \cite{nielsen2010quantum}.

Por sua vez, a simulação quântica também apresenta desafios. Devido à capacidade dos computadores quânticos de sobrepor um número exponencial de estados possíveis em relação à sua capacidade física, a simulação precisa ser capaz de armazenar uma quantia exponencial de informação, tornando o consumo de memória inviável. De forma a mitigar esse problema, diferentes técnicas de simulação surgem, tentando otimizar o consumo de memória e tempo, como a simulação esparsa \cite{jaques2022leveraging} e o método de tableau \cite{aaronson2004improved}.

Este trabalho consiste na implementação de uma dessas técnicas: o uso de diagramas de decisão para simulação de circuitos quânticos. Nessa técnica, o circuito a ser simulado tem suas portas lógicas convertidas em matrizes e a execução do circuito é representada pela multiplicação sucessiva dessas matrizes. Porém, como mencionado previamente, o tamanho da informação a ser armazenada cresce exponencialmente em relação à largura do circuito, esse crescimento é refletido nas matrizes. Por isso o QMDD (Quantum Multiple-Value Decision Diagram) é útil, essa estrutura de dados permite representar matrizes de forma concisa, explorando a redundância na composição delas \cite{fujita1997multi}. Essa técnica de compressão apresenta ganho variado, pois matrizes com maior redundância serão melhor compactadas que matrizes de menor redundância. Por exemplo, uma matriz 1000x1000 na qual todos os valores são 1 pode ser descrita exatamente dessa forma "uma matriz 1000x1000 preenchida inteiramente com o valor 1" ao invés de dedicar espaço suficiente para armazenar \num{1000000} de números 1 um ao lado do outro. Já na matriz abaixo, todos os valores são diferentes, logo o QMDD não será capaz de compactá-la com a mesma eficácia, consumindo mais memória.

\begin{equation}
  \label{distinctmatrix}
  \begin{bmatrix}
    1 & 2 & \cdots & 1000 \\
    1001 & 1002 & \cdots & 2000 \\
    \vdots & \vdots & \ddots & \vdots \\
    999001 & 999002 & \cdots & 1000000
  \end{bmatrix}
\end{equation}

No caso das matrizes relevantes para a computação quântica, a repetição de valores é muito comum por conta da natureza das operações que levam à construção delas. Por isso, o QMDD se torna uma estrutura eficiente para realizar a simulação de circuitos quânticos. 

A partir disso, será feita uma implementação dos algoritmos necessários para simulação de um circuito quântico, envolvendo a conversão de cada porta em um QMDD que representa a matriz equivalente à porta, o produto entre dois QMDDs e a medição sobre determinado QMDD. Essa implementação então será integrada à plataforma Ket \cite{evandro}, uma interface feita para facilitar a descrição de algoritmos quânticos. Então, por meio do módulo de simulação por QMDD, o usuário poderá simular de forma eficiente algoritmos que não exijam matrizes de baixa redundância. Para os casos em que o crescimento exponencial do diagrama é inevitável (casos de alto entrelaçamento entre os estados do circuito), a simulação ainda é possível, apesar de ser viável apenas para circuitos pequenos.

Além disso, não é intuitivamente claro quais circuitos levarão a uma matriz com alta ou baixa redundância. Então, este trabalho buscará comparar o desempenho de diferentes circuitos de forma a analisar se existem aspectos em comum entre os circuitos que apresentaram ganho significativo.

\section{Objetivos}

\subsection{Objetivo Geral}

Implementar um novo método de simulação para o Ket, que poderá ser escolhido pelo usuário para otimizar a simulação de circuitos cujas representações matriciais possuam grande redundância.

\subsection{Objetivos Específicos}

\begin{enumerate}[label=O\arabic*]
  \item Compilar e descrever os algoritmos necessários para o QMDD;

  \item Implementar um simulador utilizando os algoritmos do artigo;

  \item Integrar o simulador produzido ao código fonte do Ket;

  \item Identificar em quais cenários o simulador produzido apresenta vantagem em relação aos métodos de simulação já disponíveis na plataforma.
\end{enumerate}

\section {Estrutura do Documento}

No \cref{chap:compq} são introduzidos os conceitos fundamentais da computação quântica e suas diferentes formas de representações. Ao longo dessa introdução, serão pontuados os aspectos relavantes para definição dos requisitos que o QMDD deve atender e, também, os motivos pelos quais a simulação quântica não é trivial. Em seguida, são estabelecidas as funcionalidades necessárias para que uma determinada ferramenta seja suficiente para realizar a simulação de um circuito quântico. % TODO: Além disso, será fornecido um circuito quântico que será utilizado ao longo do trabalho para servir de exemplo para os algoritmos a serem apresentados.

Então, no \cref{chap:qmdd}, foi feita uma breve introdução ao funcionamento de grafos de forma a facilitar o entendimento da estrutura do QMDD, que se baseia em grafos dirigidos. Então, serão apresentadas todas as características de um QMDD, assim como a descrição formal dos algoritmos necessários para implementar as funcionalidades previamente descritas como necessárias para a simulação de um circuito quântico.

Finalmente, o \cref{chap:tcc2} tratará do planejamento das atividades para a segunda parte deste trabalho. Essa etapa consistirá da implementação dos algoritmos apresentados e da realização de experimentos utilizando o QMDD. O objetivo dos experiementos é garantir que a implementação produz resultados corretos, além de permitir a análise de desempenho desse método para simulação de circuitos quânticos e verificar em quais tipos de circuito há maior ganho de desempenho.

\section{Trabalhos Correlatos}

O uso de Diagramas de Decisão para representação de circuitos quânticos foi inicialmente proposto por \cite{miller2006qmdd}. Neste artigo, é apresentado o uso de uma estrutura baseada em diagramas de decisão para a representação concisa de matrizes que apresentam alta taxa de redundância em seus valores. Além disso, são apresentados os algoritmos necessários para manipular essas matrizes da forma necessária para realização da simulação de circuitos quânticos.

Dada a utilidade do QMDD na execução de funções quânticas, \cite{kydros2025tutorial} apresenta um manual de implementação dessa estrutura. Nesse manual, os algoritmos são definidos formalmente por meio de pseudocódigo e acompanhados das estruturas necessárias para garantir que a implementação seja eficiente. Por exemplo, a definição do uso de tabelas hash para identificação de arestas já existentes. 

\section{Metodologia}

Este trabalho apresenta a construção de um simulador a partir de um método de simulação ainda não utilizado pela plataforma Ket. A construção desse simulador será feita por meio da compilação de algoritmos existentes na literatura de forma cumprir todos os requisitos do Ket para simulação de circuitos quânticos.

Os resultados da implementação serão analisados de forma qualiquantitativa. Será feita uma comparação das estatísticas produzidas pela análise de desempenho do simulador implementado e dos simuladores já disponíveis no Ket de forma a verifcar que existem cenários nos quais o novo simulador apresenta vantagem em relação aos já existentes. Além disso, será análisado em quais cenários o novo simulador apresentou vantagem e em quais o ganho não foi significativo para identificar aspectos em comum dos algoritmos quânticos que puderam ser simulados de forma eficaz por meio do método apresentado. 
